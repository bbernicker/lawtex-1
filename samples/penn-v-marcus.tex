\documentclass[12pt]{lawmemo} 
\usepackage[margin=1in]{geometry}
\usepackage{newcent,microtype}
\usepackage{hyperref}

\renewcommand{\baselinestretch}{1.48}
\setlength{\parskip}{2.5ex plus 1ex minus .5ex}

%Declaration of the sources here
\makeandletter
\newcase{Banks}{Commonwealth v. Banks}{658\ A.2d}{752}{(Pa.\ 1995)}
\newcase{Cunha}{Cunha v. Superior Court}{466\ P.2d}{704}{(Cal.\ 1970)}
\newcase{Darling}{Darling v. State}{768\ A.2d}{463}{(Del.\ 2001)}
\newcase{Henry}{Henry v. United States}{361\ U.S.}{98}{(1959)} 
\newcase{Oden}{People v. Oden}{329\ N.E.2d}{188}{(N.Y. 1975)} 
\newcase{Ratcliff}{People v. Ratcliff}{778\ P.2d}{1371}{(Colo.\ 1989)}
\newcase{Castro}{State v. Castro}{891\ A.2d}{848}{(R.I.\ 2006)} 
\newcase{Davis}{United States v. Davis}{561\ F.2d}{1014}{(D.C.\ Cir.\ 1977)}
\newcase{Green}{United States v. Green}{670\ F.2d}{1148}{(D.C.\ Cir.\ 1981)}
\newcase{Pringle}{Maryland v. Pringle}{540 U.S.}{366}{(2003)}
\newcase{Kennedy}{Commonwealth v. Kennedy}{690 N.E.2d}{436}{(Mass.\ 1998)}
\newcase{Moore}{State v. Moore}{853 A.2d}{903}{(N.J.\ 1998)}

\begin{document}

\title{Office Memorandum}
\recipient{Frank Smith}
\author{Bill Jones}
\date{27 October 2009}
\subject{Pennsylvania v.\ Marcus}
\maketitle 
\thispagestyle{empty}


\section{Question Presented} Whether an experienced police officer has probable cause
to arrest a person when she observes him engaging in a hand-to-hand exchange of
currency for small unidentified objects in an area known to be used by drug
dealers.

%\vspace{-12pt}
%\begin{center}\textbf{Cases Cited}\end{center}
%\vspace{-18pt}
%\printindex{Case}{Cases Cites}

\section{Table of Authorities}
\printindex{Case}

\section{Brief Answer} No; a general suspicion of guilt is insufficient to
sustain probable cause for arrest.  Rather, the law requires evidence that
would lead an objective, reasonable person to believe that a crime has been
committed. Failing to demonstrate such a pattern, the arrest here is invalid.

%\vspace{-6mm} 
\section{Facts}  On 4 May 2006, Philadelphia Police Officer
Yolanda Bello was patrolling undercover in the residential Kensington
neighborhood in north Philadelphia, as part of her recent assignment to the
Narcotics Strike Force. At 11:00 AM on that day, Bello observed Ike Marcus from
a distance of approximately one city block.  She had not seen Marcus before,
and Marcus had no prior record or known interaction with police.  Marcus was
walking, alone, towards Bello on the opposite side of the road.  He stopped in
front of an apartment building to speak with an unidentified man standing on
the street.  Marcus handed the man a single bill of currency.  The man handed
Marcus several small, unidentified objects, which he placed in his shirt
pocket.  The two men parted. Marcus turned and walked in the other direction,
while the man entered the apartment building.

These actions by Marcus and the unidentified male formed the entirety of the
direct evidence leading to arrest.  Bello testifies that she immediately
suspected that she had witnessed a drug transaction.  She followed Marcus and
performed an arrest; during the subsequent search, she found two vials
containing crack cocaine.  She reported that Marcus performed no actions
between the transaction and arrest to further increase suspicion, nor did she
fear for her own safety.  

\section{Discussion} 

\subsection{The Constitutional Standard.} This court has specified that ability
for law enforcement to perform an arrest without a warrant is ``restricted to
offenses committed in their presence, or where they have `reasonable grounds'
to believe that the person to be arrested has committed or is committing a
felony.'' \pincite{Henry}{100}. Expounding upon the concept of ``reasonable
grounds,'' the court clarified that ``common rumor or report, suspicion, or
even ‘strong reason to suspect’ was not adequate to support an arrest.''
\pincite{Henry}{101}.

While the level of evidence required to establish guilt at trial is not
necessary, good faith on the part of the arresting officers is insufficient.
Rather, ``[p]robable cause exists if the facts and circumstances known to the
officer warrant a prudent man in believing that the offense has been
committed.'' \pincite{Henry}{102}. This implies an objective standard for
evaluation of probable cause.  However, as the Court has stated, this standard
``is incapable of precise definition or quantification into percentages'' and
therefore ``depends on the totality of the circumstances''. \pincite{Pringle}{371}.

\subsection{Dispositive Factors for Probable Cause.} In applying the standard to
the sort of transaction in question, courts have first looked for clear
evidence that, standing on its own, is dispositive of the question of reasonable
grounds.  Upon finding examples of such factors, the inquiry is complete.
\Seeeg~\cite{Castro} (observation of cocaine during a traffic
stop was sufficient to arrest both parties to the immediately preceding
transaction); \cite{Davis} (multiple suspicious transactions and visible pink
pills consistent with illicit substances provided probable cause for arrest);
\cite{Darling} (multiple transactions and the use of masks to conceal the
sellers' identities provided probable cause).

In a case where an arrest was invalidated, the Pennsylvania Supreme Court
specifically listed the absence of such potential dispositive factors as:
observation of drugs or containers commonly known to hold drugs, observation of
``multiple, complex, suspicious transactions,'' and ``a citizen's or
informant's tip.'' \pincite{Banks}{455}.  Rather, in \textit{Banks}, the
grounds for arrest were a rapid, on-street transaction similar to that in
\textit{Marcus}, as well as the suspect's flight to avoid arrest.  The court pointed
out that neither the transaction, nor flight, standing by themselves would be
dispositive of criminal activity.  Further, the fact of the on-street
transaction ``cannot be added to, or melded with the fact of flight'' to form
probable cause. \pincite{Banks}{456}.

\subsection{Probable Cause Inferred from the Totality of Circumstances.} In
contrast, \cite{Green}, provides an example where probable cause was
established, not from individual factors that alone were dispositive, but with
guilt inferred though the totality of the circumstances.  Appellant Green
contested his conviction for possession of heroin with intent to distribute.
An officer watching from a concealed location observed an unidentified man
perform a purchase of an unknown object through the use of a third-party
intermediary.  The intermediary received the money from the buyer, delivered
the money to Green in exchange for the unknown merchandise, and returned to the
buyer with the merchandise.  

The officer described this as consistent with a ``two-party drug transaction,''
in which two individuals sell the drugs to give themselves some protection from
robbery.  \See~\pincite{Green}{1151~n.1}.  Green made attempts to
conceal the large paper bag in which the object was kept, by stuffing it into a
pants pocket.  When approached by police officers in an unmarked car, Green
made efforts to evade the officers, as well as to dispose of the bag in a
nearby building.  Green was then arrested; the bag was found to contain heroin.

In its analysis, the court recognizes that (as in~\textit{Banks}) the individual
factors are insufficient by themselves to establish probable cause.  However,
unlike in \textit{Banks}, the overall situation warranted the
arrest, explaining:  \begin{quote} [a]lthough none of these four factors is adequate by itself to
establish probable cause, it is their combination in the particular
circumstances confronting [the officers] that is the proper subject
of consideration.  Probable cause is not determined by observing some single
factor which this court has deemed relevant, or even by observing any certain
number of them.  Rather, probable cause exists if the totality of the
circumstances as viewed by a reasonable and prudent police officer in light of
his training and experience, would lead that police officer to believe that
criminal offense has been or is being committed.  \end{quote}

\noindent \pincite{Green}{1152} (footnote omitted).  The court recognized
that this series of suspicious actions, viewed by an officer with the
experience to recognize the nature of the actions, could only be viewed in the
context of a drug transaction.   ``No plausible, innocent explanations for this
sequence of behavior readily spring to mind\ldotss'' \pincite{Green}{1153}.
Note the qualification of the \textit{Henry} standard (that of the ``prudent
man'') with the more specific ``prudent police officer.'' This does not claim
to relieve the police officer of the \textit{Henry} requirement to meet an
objective standard of probable cause; rather it provides that when assessing
reasonableness, the unique knowledge available to the officer is.  to be
considered.  However, even when invoking ``experience,'' the officer must
demonstrate that the conclusions following from her experience were objectively
reasonable. 

\subsection{Example Applications of the Standard.} Consequently, courts have routinely
invalidated arrests when the arresting officers are unable to provide clear
reason to assume criminal behavior, despite the fact of significant field
experience of the arresting officer.  In \cite{Ratcliff}, officers observed
an exchange between the defendant and a known drug dealer outside a bar.  The
defendant was arrested and searched, revealing cocaine.   The court invalidated
the arrest, stating that an objective view of the transaction consisted of
nothing more than ``a brief exchange of some object or objects outside a bar.''
\pincite{Ratcliff}{1377}.  Despite the officers' familiarity with the area as
one known for drug transactions, ``it was nonetheless a public way and people
obviously had many reasons for being there other than drug dealing.''
\pincite{Ratcliff}{1378}.  \Seealso \pincite{Cunha}{708}
(``Transactions conducted by pedestrians are not per se illegal, and the
participants' apparent concern with privacy does not imply guilt.'');
\cite{Oden} (holding on-street exchange of an envelope insufficient 
for probable cause). 

The courts have upheld arrest on a totality-of-circumstances standard with less
evidence than that of \textit{Green}.  While they relied heavily on officer
experience to interpret a pattern of behavior, they do not endorse the sole
dependence on the officer's judgment, as Petitioners seek to allow.  In
\cite{Kennedy}, the officer witnessed a known drug-dealer approach a car to
speak briefly with the occupant, run away, and return shortly to the vehicle to
make an exchange.  The officer interpreted the short interlude as an
opportunity for the dealer to retrieve drugs from his ``stash\ldotss
warrant[ing] the officer to conclude that he was observing a classic street
level drug transaction.'' \pincite{Kennedy}{439~n.2}.  In \cite{Moore}, a
similar pattern of leaving and returning, followed by a transaction for
unidentified objects, also provided probable cause for arrest.


\subsection{Applying the Standard to \textit{Marcus}.} It seems clear that
\textit{Commonwealth v.\ Marcus} is absent factors that are clearly dispositive of
probable cause.  Thus probable cause, if it exists, must be inferred from the
totality of the circumstances.  In the cases with valid arrests that result from
such an analysis (\textit{e.g.} \textit{Green}, \textit{Kennedy}, and
\textit{Moore}), interpretations of the pattern of behavior were provided that
could allow a reasonable person to believe that a crime was committed---the
\textit{Henry} standard.  No such rationale has been provided
in \textit{Marcus}, which would have allowed an evaluation of the
reasonableness of that inference.  Attempting to find an inference from the
limited evidence available, it would seem that an objective analysis could at
most support a finding of ``a brief exchange of some object or
objects,'' \pincite[n]{Ratcliff}{1377}, as the \textit{Ratcliff} court
characterized.  Therefore, without evidence indicative of a crime, I conclude
that probable cause did not exist.

\section{Conclusion} In this memorandum, I have evaluated a claim of probable
cause for arrest in a suspected drug sale, in which the evidence consisted only
of a rapid, on-street transaction for unidentified objects.  After first
identifying the constitutional standard for warrantless arrest as provided in
\textit{Henry v.\ United States}, which requires evidence that would lead an
objective observer to conclude that a crime was committed, I examined
applications of this standard by lower courts.  The courts have expected a
criminal explanation for an observed pattern of behavior; one which may be
elucidated by the benefit of an officer's experience, but may not rest upon
confidence alone.  Such an on-street transaction, without further evidence,
would not objectively lead to the inference of criminal behavior.  Therefore, I conclude that the
arrest in this case was invalid.

\end{document}
