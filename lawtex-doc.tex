\documentclass[letterpaper]{article}
\usepackage[margin=1in]{geometry}
\usepackage[]{bluebook}
\usepackage{shortvrb,fancyvrb,setspace}
\usepackage{newcent,microtype,hyperref}
\usepackage{xparse}
\usepackage{trace}

\RequirePackage{xr} % used in bluebook.sty for addReference
\externaldocument{sample-appendix} % used in bluebook.sty for addReference

\let\articleMaketitle=\maketitle
\let\articleSection=\section
\let\articleSubsection=\subsection
\let\articleSubsubsection=\subsubsection
\let\articleThesection=\thesection
\let\articleThesubsection\thesubsection

\makeatletter
\makeandletter
\def\SuppressClass{}
\input lawbrief.cls\relax

\def\LawTeX{%
L\kern-.36em
{\setbox0=\hbox{T}%
\vbox to \ht0{\hbox{\the\scriptfont0 A}\vss}}%
\kern-.15em {\hbox{\the\scriptfont0 W}}\kern-.3em\TeX
}

\MakeShortVerb{_} %that's an underscore
%\catcode9=13	 %that's a tab 
%\def	{\hspace*{1.5em}} %ditto
\renewcommand{\arg}[1]{\texttt{\{}\emph{#1}\texttt{\}}}
\newcommand{\opt}[1]{\texttt{[}\emph{#1}\texttt{]}}

%========================================================
\def\Example{%
\catcode`\^^M=\active
\parindent=0pt

Example: 
\vspace{-7pt}
\VerbatimEnvironment
\catcode`\^^M=5\begin{VerbatimOut}[codes={\catcode`\^^a3=12\catcode`\^^a7=12\catcode`\^^b5=12\catcode`\^^b6=12}]{\jobname.tmp}}

%========================================================
\def\endExample{%
\end{VerbatimOut}%
\VerbatimInput[gobble=0,commentchar=^^a3,commandchars=^^a7^^b5^^b6,numbersep=3pt]{\jobname.tmp}%
\catcode`\^^a3=9\relax%
\vspace{-6pt}%
Results (Normal Mode): 
\vspace{6pt}

\let\@parskip=\parskip
\parskip=0pt
\leftskip=1.5em
\input{\jobname.tmp} 
\par

\leftskip=0em

\vspace{6pt}
\noindent Results (Law-Review Mode):%
\nopagebreak

\vspace{5pt}%
\begingroup%
\@bbSetLawReview%
\xdef\@bbLastSource{}%
\def\newmisc##1##2{\setboolean{##1@FirstUse}{true}}%
\def\newstatute##1##2{\setboolean{##1@FirstUse}{true}}%
\def\newbook##1##2##3##4{\setboolean{##1@FirstUse}{true}}%
\def\newincollection##1##2##3##4##5##6{
	\begingroup
		\def\textit####1{####1}
		\def\emph####1{####1}
		\xdef\@HandleName{##1}
	\endgroup
	\setboolean{\@HandleName @FirstUse}{true}
	\setboolean{##4@FirstUse}{true}}%
\def\newinsingleauthorcollection##1##2##3##4##5##6{
	\begingroup
		\def\textit####1{####1}
		\def\emph####1{####1}
		\xdef\@HandleName{##1}
	\endgroup
	\setboolean{\@HandleName @FirstUse}{true}
	\setboolean{##4@FirstUse}{true}}%
\def\newarticle##1##2##3##4##5##6{%
	\begingroup%
		\def\textit####1{####1}%
		\def\emph####1{####1}%
		\xdef\@HandleName{##1}%
	\endgroup%
	\setboolean{\@HandleName @FirstUse}{true}}%
\def\footnote##1{%
	\setboolean{@bbInFootnote}{true}%
	\stepcounter{footnote}\textsuperscript{\thefootnote} ##1%
	\setboolean{@bbInFootnote}{false}%
}%
\input{\jobname.tmp}
\parskip=0pt%
\leftskip=1.5em%


\par
\endgroup}

%\def\Example{%
%\catcode`\^^M=\active
%\@ifnextchar[{\catcode`\^^M=5\Example@}{\catcode`\^^M=5\Begin@Example}}
%
%\def\endExample{%
%\end{VerbatimOut}%
%\Below@Example{\input{\jobname.tmp}}
%\Below@LRExample{\input{\jobname.tmp}}}
%
%%========================================================
%\renewcommand{\Begin@Example}{%
%\parindent=0pt
%\VerbatimEnvironment
%
%\noindent Example: \vspace{-12pt plus 0pt minus 0pt}%
%\begin{VerbatimOut}[codes={\catcode`\^^a3=12\catcode`\^^a7=12\catcode`\^^b5=12\catcode`\^^b6=12}]{\jobname.tmp}}
%
%%========================================================
%\renewcommand{\Below@Example}[1]{%
%\VerbatimInput[gobble=0,commentchar=^^a3,commandchars=^^a7^^b5^^b6,numbersep=3pt]{\jobname.tmp}%
%\catcode`\^^a3=9\relax%
%\noindent Results: 
%
%\let\@parskip=\parskip
%\parskip=0pt
%\leftskip=1.5em
%#1%
%
%\let\parskip=\@parskip
%\leftskip=0em
%~
%
%}

%========================================================
\def\StripCmds#1{%
	\begingroup%
		\def\textit##1{##1}%
		\def\textsc##1{##1}%
		\def\emph##1{##1}%
		\xdef\@HandleName{#1}%
	\endgroup%
	\@HandleName%
}
	
%\newcommand{\Below@LRExample}[1]{%
%\catcode`\^^a3=9\relax%
%\noindent Results (Law-Review Mode): \par ~\par 
%
%\parskip=0pt
%\leftskip=1.5em
%\begingroup
%\@bbSetLawReview
%\xdef\@bbLastSource{}
%\def\newmisc##1##2{\setboolean{##1@FirstUse}{true}}%
%\def\newstatute##1##2{\setboolean{##1@FirstUse}{true}}%
%\def\newbook##1##2##3##4{\setboolean{##1@FirstUse}{true}}%
%%
%\def\newincollection##1##2##3##4##5##6{
	%\begingroup
		%\def\textit####1{####1}
		%\def\emph####1{####1}
		%\xdef\@HandleName{##1}
	%\endgroup
	%\setboolean{\@HandleName @FirstUse}{true}
	%\setboolean{##4@FirstUse}{true}}%
%%
%\def\newarticle##1##2##3##4##5##6{
	%\begingroup
		%\def\textit####1{####1}
		%\def\emph####1{####1}
		%\xdef\@HandleName{##1}
	%\endgroup
	%\setboolean{\@HandleName @FirstUse}{true}}%
%%
%\def\footnote##1{%
	%\setboolean{@bbInFootnote}{true}%
	%\stepcounter{footnote}\textsuperscript{\thefootnote} ##1%
	%\setboolean{@bbInFootnote}{false}%
%}%
%#1%
%\par
%\endgroup}

%=========================================================
\DeclareDocumentCommand{\cmd}{m O{} G{} G{} G{} G{} G{} G{} o}{%
\addvspace{16pt}
\noindent\texttt{\large\bf\textbackslash{}#1} --- #9 

\penalty10000
\noindent Usage:

\penalty10000
\texttt{\textbackslash{}#1}\ifempty{#2}{}{[\emph{#2}]}\ifempty{#3}{}{\{\emph{#3}\}}\ifempty{#4}{}{\{\emph{#4}\}}\ifempty{#5}{}{\{\emph{#5}\}}\ifempty{#6}{}{\{\emph{#6}\}}\ifempty{#7}{}{\{\emph{#7}\}}\ifempty{#8}{}{\{\emph{#8}\}}}
%==========================================================

\title{Law\TeX{}: Automated \LaTeX{} Legal Citations} 
\author{Christopher De Coro}
\date{October 7, 2014}

\citecase{Lochner v. New York, 198 U.S. 45 (1905)}

\let\section=\articleSection
\let\subsection=\articleSubsection
\let\thesection=\articleThesection
\let\thesubsection=\articleThesubsection

\begin{document}
\singlespacing
\frenchspacing
\articleMaketitle

Lawyers that value high-quality typography in their work have a significant problem when they try to use \LaTeX{}: no support for legal citations. 
The \LawTeX{} package remedies this. Define a source, \texttt{\textbackslash{}citecase\{Lochner v. New York, 198 U.S. 45 (1905)\}}, and cite it using \texttt{\textbackslash{}cite\{Lochner\}}, \cite{Lochner}. Pincites are also supported, \texttt{\textbackslash{}see \textbackslash{}pincite\{Lochner\}\{51\}}, \see\pincite{Lochner}{51} (note that the second citation was converted automatically to \id.). 

\section*{Contents}
\begingroup
\renewcommand*\l@section[2]{%
    \addpenalty\@secpenalty%		Prefer to break before or after the tocline, not during
    \setlength\@tempdima{1em}%		Size of the space reserved for section number (which isn't there, so 0)
	{
      \parindent=0pt%				Dont automatically indent
	  \@tocline{{#1}}{#2}
	}
}
\tableofcontents
\endgroup

\parskip=8pt

\section{Introduction}
I come to the legal world after a long time in Computer Science academia. In Computer Science, \LaTeX{} using Bib\TeX{} is the unchallenged standard for typesetting anything important. We would never dream of manually adding and changing reference numbers and citation formats. When you change a block of text, say, by moving one paragraph in front of the other, we expect that all numbered references will be automatically updated to match their new ordering. It seems \emph{obvious} that this is a task for machines, and not people. 

Yet once I made this career change, I was taken aback at the primitive state of automated citations in the legal field. Despite having a \emph{drastically} more complicated citation system (and even the term ``drastically,'' used to describe the difference between the non-existent CS citation system and the Bluebook seems like an understatement) all changes were done manually. So after making any changes in your document, and before submitting, the legal practitioner must make a long pass through the entire document, asking, \emph{inter alia}\footnote{Actually, I don't care for the use of Latin unless it is a term of art to describe something more succinctly; as in the way ``res ipsa loquitor'' is a more convenient expression than ``we can infer negligence from the fact that the instrument of harm was in the exclusive control of the defendant and would not ordinarily cause harm in the absence of negligence.'' But despite the fact that ``among others'' is nearly as short, lawyers seem to love ``\emph{inter alia}'' all the same.}:

\begin{itemize}
\item Did I cite this already, such that I should use a ``short cite'' here instead of a ``long cite''?  
\item This short cite is now a long cite; what was that starting page number again?
\item In a law review article, was this source cited within the last 5 footnotes such that I need a full citation? 
\item Was the previous source immediately preceding this one, such that I should use \emph{id.}?
\item Was the previous citation not only to the same source but the same pincite, such that no additional pincite is needed?
\item Did I remove a signal, such that this \emph{id.} is now capitalized?
\item I deleted the last cite on a page to a particular source, so my former \emph{passim} now needs numbers again. What pages were there? 
\item And perhaps worst of all, is this cite on a different page than before, such that I need to update the Table of Authorities?
%\item ---Volume number, etc.---
\end{itemize}

Some of these seem trivial, of course. We get quite good at scanning through and applying Bluebook rules after extended practice. But that fact that there is \emph{any} work for us to do after a change means that we must re-read through it all to make sure that it is correct---we can't trust the machine to do this automatically.  

\section{A Simple Example}
All that's really necessary to use \LawTeX{} is the directive \verb+\usepackage{bluebook}+ in the document preamble. There are also a few convenient document classes that are provided that you can use; some modifications to the class files will likely be necessary to fit your jurisdiction (and they're missing parameters to control some of the information). The classes are \verb+lawbrief.cls+ for standard briefs (with a table of authorities), and a simple \verb+lawmemo.cls+ for legal memos. 

For now, consult the file {\tt postal-tro-motion.tex} in the {\tt samples/} folder for an example. This is the most thorough example, with {\tt hosanna-tabor.tex} another good example. You can compile it with the following commands (the _makeindex_ commands are not necessary if _\write18_ is enabled, as this is done automatically by the package):

\begin{verbatim}
pdflatex postal-tro-motion
makeindex -s ../lawcitations.ist -r Case.idx
makeindex -s ../lawcitations.ist -r Statute.idx
pdflatex postal-tro-motion
\end{verbatim}

There are additional files in the {\tt samples/} directory. LaW\TeX{} has two modes: Normal Mode (that is, for court submissions and most other legal documents) and Law Review Mode. As it's name would suggest, Normal Mode is the default; to select Law Review Mode, pass the _lawreview_ option to the _bluebook.sty_ or the class file. I have provided each of the sample files in both Normal Mode and Law Review Mode, in order to demonstrate the difference. Both are generated from the same file, and the only difference is the package option. However, note that I have had to use some conditional code to switch between both modes in a single file; these can be safely ignored in your own files, assuming that you do not need to switch back and forth.

\makeatletter
\parskip=6pt
\parindent=1.5em
\leftskip=0em
\frenchspacing
\font\parasymbolfont=phvr at 10pt
\font\sectionsymbolfont=pncr at 10pt
 %\RequirePackage{xr} % in preamble
 \section{Defining Sources}\vspace{-8pt}

 \citecase[Steel Seizure Case]{Youngstown Sheet & Tube Co. v. Sawyer, 343 U.S. 579 (1952)}
 \citecase[Menard]{Menard, Inc. v Escanaba, 315 Mich. App. 512; 891 N.W.2d 1 (2016)}
\cmd{citecase}[Short Name]{Standard Case Citation}[Defines a new case citation, using standard citation form.]
 \begin{Example}
   %\citecase{Lochner v. New York, 198 U.S. 45 (1905)}
   %\citecase[Steel Seizure Case]{Youngstown Sheet & Tube Co. v. Sawyer, 343 U.S. 579 (1952)}
   \cite[l]{Lochner}. \\ 
   \pincite{Lochner}{48}. \\
   \pincite{Steel Seizure Case}{602}. \\
   \pincite{Lochner}{52}. \\
   
   %\citecase[Menard]{Menard, Inc. v Escanaba, 315 Mich. App. 512; 891 N.W.2d 1 (2016)} \\
   \pincite{Menard}{513; 2} \\ % will cite both reporters. 
   \pincite{Menard}{\_; 2} \\ % will cite both reporters for long citations, but only the second reporter for short cites. The single \_ indicates that the writer does not know where the quote is in the first reporter. If this is a short cite only, the first reporter will be skipped and only the second reporter will appear. If you want to force appearance of the first reporter, just use something different than a single \_, a \_\_ will work. This hack accommodates situations where the official reporter is not available, such as when only a copy of the unafficial reporter is available online.
 \end{Example}
 \noindent In order to cite legal sources in the body of the text, first include a _\citecase_ command with the 
 case citation to define the case as a source. At the point in the text where the citation should appear, use
 _\cite_ or _\pincite_, passing the short name as an argument.  Spacing and commas are important (extra spaces will be preserved). 

 By default, the first party is used for the short name of the case, unless that party is 
 ``United States,'' ``State,'' ``Commonwealth,'' or ``People,'' in which case the second party is used.  
 To override the choice of short name, set the optional parameter. 

 With case citations that differ from the standard format, you may need to use the _\newcase_ command.
 For example, in ``\emph{Marbury v. Madison}, 5 U.S. (1 Cranch), 137 (1803)'', the extra parentheses around
 the ``(1~Cranch.)'' will throw off the parser. See the way that this is cited below. 

 \cmd{addReference}{[AppendixName]}{ShortName}{reference name}[Links a case to an appendix record. To use this with an external appendix, put in the preamble: \\RequirePackage{xr} and \\externaldocument{appendix-file-name}.]
 \citecase[Patru]{Patru v City of Wayne, unpublished per curiam opinion of the Court of Appeals, issued May 8, 2018 (Docket No. 337547)}
 \addReference[sample-appendix]{Patru}{patruvwayne}
 \begin{Example}
   %\RequirePackage{xr} % in preamble
   %\externaldocument{sample-appendix} % in preamble
   %\citecase[Patru]{Patru v City of Wayne, unpublished per curiam opinion of the Court of Appeals, issued May 8, 2018 (Docket No. 337547)}
   %\addReference[Sample Appendix]{Patru}{patruvwayne}
   \pincite{Patru}{3}
 \end{Example}
 \noindent The _\caseReference_ command links a case to an appendix. Call it after you've defined a case. Whenever you cite to the case, a reference will be made to the right page in the Appendix. In Michigan, unpublished cases must be provided. This command makes it easy for people to look up the case in the appendix.

 
\cmd{newcase}{Short Name}{Full Name}{Reporter}{Starting Page}{Parenthetical}[Defines a new case citation.]
 \newcase{Marbury}{Marbury v. Madison}{5 U.S. (1 Cranch)}{137}{(1803)}
 \begin{Example}
   %\newcase{Marbury}{Marbury v. Madison}{5 U.S. (1 Cranch)}{137}{(1803)}
   \pincite{Marbury}{140}.
 \end{Example}

 \noindent The _\newcase_ command is equivalent to the _\citecase_ command (which in fact calls _\newcase_ internally). 
 Generally, it is simpler to use the _\citecase_ command, but _\newcase_ is necessary for case citations that differ from the usual, 
 such as \textit{Marbury}, above, because of the parentheses around the ``(1 Cranch)''. 

% Actually invoke the commands for the documentation:


\cmd{newbook}{Short Name}{Authors}{Title}{Parenthetical}[Define a new book citation.]
 \begin{Example}
   \newbook{Prosser and Keaton}{William Lloyd Prosser & W. Page Keaton}
     {The Law of Torts}{(2nd ed., 1953)} 
   \newbook{Schelling}{Thomas Schelling}
     {A Process of Residential Segregation: Neighborhood Tipping, 
     {\upshape\it reprinted in} Economic Foundations of Property Law {\upshape 307,}}
     {(Bruce A. Ackerman ed., 1975)}
   \pincite{Schelling}{308} \\
   \pincite{Prosser and Keaton}{vol. 2, 15}. \\
   \pincite{Prosser and Keaton}{vol. 2, 345}. \\
   \pincite{Prosser and Keaton}{vol. 4, 876}. \\ 
   \pincite{Schelling}{310}.
 \end{Example}
 
 \noindent This command introduces a new book citation. Such citations are first written with a 
 long name that includes [volume] author, title, citation and parenthetical. Subsequent invocations will use the short name given by the first argument,
 which is also used as the argument to _\cite_, followed by \textit{supra}. In law-review mode, this is further followed by ``note n,'' the first 
 footnote in which the article was cited. Note that any _\textit_ or _\emph_ and their braces are stripped out of the short name (see _\newarticle_ for an example). 

 To cite a particular volume, use a pincite in the form _\pincite_\arg{Short Name}\{vol. 1, 123\}. That is, it must start with ``vol.''
 followed by exactly one space, then the number followed by a comma and another space. 

 This command is also used for a \textit{reprinted in} citation. Note the use of _\upshape_ and _\it_ (or equivalent) that is necessary
 to ensure the correct formatting.


\cmd{newarticle}{Short Name}{Authors}{Title}{Journal}{Start Page}{Parenthetical}[Define a new article citation.]
 \begin{Example}
   \newarticle{Note, \textit{The Ministerial Exception}}
     {Note}{The Ministerial Exception To Title VII}
     {121 Harv. L. Rev.}{1776}{(2009)}
   \newarticle{Ward}{Barbara Ward}{Progress for a Small Planet}
     {Harv. Bus. Rev.}{Sept.--Oct. 1979, at 89}{}
   \cite[l]{Note, The Ministerial Exception}.  \\
   \pincite{Ward}{90}. \\
   \pincite{Note, The Ministerial Exception}{1800}. 
 \end{Example}

 \noindent This command introduces a new law-review article-type citation. Such citations are first written with a 
 long name that includes author, title, citation and parenthetical. Subsequent invocations will use the short name given by the first argument,
 which is also used as the argument to _\cite_, followed by \textit{supra}. In law-review mode, this is further followed by ``note n,'' the first 
 footnote in which the article was cited. Note that any _\textit_ or _\emph_ and their braces are stripped out of the short name. 

 Note that in law review mode, the journal name is by default set in \textsc{Small Caps}, which is the Bluebook standard for 
 consecutivly-paginated journals. To produce standard type, use the
 form of _{{\upshape Harv. L. Rev.}}_ when defining the citation. 


\cmd{newincollection}{Short Name}{Authors}{Article Title}{Collection Title}{Page}{Parenthetical}[Defines a new article/chapter-in-collection citation.]

 \begin{Example}
   \newincollection{Allen, \textit{Oration}}{John Allen}
     {Oration Upon The Beauties Of Liberty}
     {Political Sermons of the American Founding Era}{vol. 1, 58}
     {(Ellis Sandoz ed., 1991)}
   \newincollection{Mather}{Moses Mather}{America's Appeal To The Impartial World}
     {Political Sermons of the American Founding Era}{vol. 1, 103}
     {(Ellis Sandoz ed., 1991)}

   \pincite{Allen, Oration}{62}. \\
   \pincite{Mather}{103}. \\
   \pincite{Allen, Oration}{78}. \\
   \pincite{Mather}{119}. \\
 \end{Example}

 Note that the parenthetical will be placed after the collection title, and therefore not printed if the collection itself is cited a second time. Note additionally
 that if the parenthetical changes across definitions for multiple documents in the collection, whichever source is actually cited first defines the parenthetical that is used.
 

\cmd{newinsingleauthorcollection}{Short Name}{Author}{Title}{Collection Title}{Page}{Parenthetical}[Defines a citation to a single-author collection.]

 \begin{Example}
   \newinsingleauthorcollection{Holmes}{Oliver Wendell Holmes}
     {Law in Science and Science in Law}{Collected Legal Papers}
     {210}{(1920)}
   \pincite{Holmes}{vol. 1, 120}. \\
   \pincite[s]{Holmes}{vol. 1, 133}. 
 \end{Example}

 \noindent This command is similar to _\newincollection_, for collections whose documents are all from the same author. The Bluebook specifies that these should
 be cited like a book, and therefore with the volume number before the author, and the author's name in small caps.


\cmd{newstatute}[optional handle@index-place]{Short Name}{Parenthetical}[Define a new statute citation.]
 \begin{Example}
   \newstatute{42 U.S.C.}{(2006)}
   \newstatute{Administrative Procedure Act}{(2006)}
   \pincite{Administrative Procedure Act}{\S 1, 5 U.S.C. \S 551}. \\
   \pincite{Administrative Procedure Act}{\S 2}. \\
   \pincite{42 U.S.C.}{\S 1983}.
 \end{Example}


 Add position indexing and separate shortnaming.

1) The references in the Statutes index can appear in the wrong order or
 2) can involve formatting like small caps (\textsc) which cannot be used as
 the internal TeX handle. This change solves both problems. A new
 optional argument is added to \newstatute. The argument can take the
 following forms:

 \newstatute[AA@]{MCL}{} - put the MCL statute at place AA in the index.

 \newstatute[am1]{\textsc{U.S. Const.} amend. I} - use am1 as the handle
 for the first ammendment. (The \textsc prevents the name from being used
 as a handle.)

 \newstatute[1@amendment1]{\textsc{U.S. Const.} amend. I} - use amendment1 as the
 handle for the first ammendment, put it at position 1 in the index.

\cmd{newmisc}{Short Name}{Full Name}[Define a general source by explicitly providing long and short citations.]
 \begin{Example}
   \newmisc{Bill of Rights 1689}{Act Declaring the Rights and Liberties of the 
     Subject and Settling the Succession of the Crown (Bill of Rights), 
     1 W. & M., sess. 2 c. 2\pin{, }{} (1689)}
   
   \pincite{Bill of Rights 1689}{\S 2}. \\
   \pincite[s]{Bill of Rights 1689}{\S 3}.
 \end{Example}

 \noindent The long cite may have the command _\Pin_\arg{Text Before}\arg{Text After}, which in the case of 
 a pincite, will insert the cite at the given location, surrounded by the text as indicated.
\section{Citing Sources}

\noindent\texttt{\large\bf\textbackslash{}[pin]cite} --- Cite a legal source using Bluebook style

\noindent Usage: 

 \texttt{\textbackslash{}cite}[\textit{Formatting}]\{\textit{Short Name}\} \\
 \hspace*{\parindent}\texttt{\textbackslash{}pincite}[\textit{Formatting}]\{\textit{Short Name}\}\{\textit{Pin Page}\} 

 \begin{Example}
   \cite[l]{Marbury}. \\ 
   \pincite[i]{Marbury}{117}. \\
   \pincite[s]{Marbury}{117}. \\
   \pincite{Steel Seizure Case}{602}. \\
   \pincite{Marbury}{122}. \\
   \See \pincite[n]{Steel Seizure Case}{625}. \\
   \Cf \pincite[s]{Steel Seizure Case}{625}.
 \end{Example}
	 
 \noindent The optional first argument forces a particular citation form, which is useful where the correct form cannot be determined automatically
 (\textit{e.g.}, the rule that one may not use \Id. in the next citation after a string cite), or that one does not capitalize id. 
 when it appears in the middle of a sentence (although when using the signal macros _\See_, _\Seealso_, etc.) this will be
 handled automatically. These options consist of a single letter, from the list as follows:  

 \noindent
	_l_ - Force long form citation, regardless if the source has appeared before. \\
	_s_ - Force short form citation, even if this is the first cite to this source. \\
	_n_ - Force reporter and page number-only citation (for cases only). \\
	_I_ - If and only if ``\textit{id.}'' is used, force it to be capitalized. \\ 
	_i_ - If and only if ``\textit{Id.}'' is used, force it to be non-capitalized. \\
	_!_ - Record a cite at this location (and thus to the ToA / record Supra), but do not actually print anything. \\ 
	_*_ - Print the citation here, but do not record it to the table of authorities.  


\cmd{Id, \textbackslash{}id}[Pin Page][Cite the previous source]
 \begin{Example}
   \pincite{Marbury}{140}. \\
   \See \id[140]. \\
   \Id[141].
 \end{Example}

 \noindent The effect is to repeat the previous cite, including the previous pin page (unless the optional argument is used
 to cite a different page). _\Id_ should generally be followed by a period or other punctuation, as the trailing period 
 will not be added automatically.

\cmd{citetext}{Arbitrary text}[Automatically place law review citations in footnotes.]

 \noindent The purpose of this command is primarily to work with automatic footnotes in law review mode. 
 Whatever text is passed as its argument will automatically be put in a footnote, if it is not already in 
 a footnote (in which case _\citetext_ does nothing). If you are not using law review mode, this command 
 is not necessary. You would ordinarily put a space between the closing punctuation and the _\citetext_,
 the macro will automatically remove that space, if appropriate.

\cmd{citeclause}{Arbitrary Text}[Cite arbitrary text in an intra-sentence citation clause]

\noindent Example:

 \begin{quote}\texttt{It is the role of the judicial department to say what the law is \\ \textbackslash{}citeclause\{\textbackslash{}see \textbackslash{}cite\{Marbury\}\} and the present case is no exception.}\end{quote}

\noindent Result (Normal Mode):

 \begin{quote}It is the role of the judicial department to say what the law is, \see \cite[s]{Marbury}, and the present case is no exception.\end{quote}

\noindent Result (Law Review mode):

 \begin{quote}It is the role of the judicial department to say what the law is,\footnote{\See \cite[s]{Marbury}} and the present case is no exception.\end{quote}

 \noindent This function is probably not necessary for production use, but allows the samples to have one source code for both
 normal and lawreview mode. When using a citation clause in the middle of a sentence, use _\citeclause_ at the location,
 WITHOUT any surrounding punctutation. In standard mode, _\citeclause_ will add surrounding commas, unless the following
 character is a period, in which case it add a preceding comma and leaves the period to follow, as-is.

 In lawreview mode, _\citeclause_ will insert a preceding comma, and insert a footnote with the cited text immediately 
 after the comma, with no punctuation following -- UNLESS the following character is a period. In the latter case, 
 _\citeclause_ will add only a preceding period, with the footnote immediately following. Also, if the first token of the 
 citation clause is one of %% the pre-defined lowercase citation signals (_\see, \cf_ etc.), it will be automatically 
 converted to the uppercase equivalent.

 Note that in either case, _\citeclause_ is not able to properly handle a citation for a quote, in which case the punctuation
 should go inside the closing ''. In that case, you will either need to make the change manually when changing between
 normal and lawreview modes, or use the _\PeriodOrComma_ macro. The latter is a Period in lawreview, and a Comma normally. 

 If you're thinking this is more hassle than its worth, you're right. Feel free to just write out the correct form manually.

%\cmd{@autofootnote}{Put argument in a footnote, if in lawreview mode}

~\par

\noindent
\begingroup\raggedright\hyphenpenalty=10000\hangindent=1.5em
\texttt{\bf \textbackslash{}Reporter, \textbackslash{}ShortName, \textbackslash{}FullName, \textbackslash{}StartPage, \textbackslash{}Parenthetical, \textbackslash{}Authors, \textbackslash{}BookTitle, \textbackslash{}SrcType, \textbackslash{}SupraNote, \textbackslash{}LastNote} \par\noindent
\endgroup
 Usage: \\
	_\Reporter_\arg{Short Name}, _\StartPage_\arg{Short Name}, \emph{etc.}

\begin{Example}
   \FullName{Steel Seizure Case} \\
   \Reporter{Marbury} \\
   \StartPage{Marbury} 
\end{Example}

 \noindent These functions are predominantly helper functions used elsewhere in the code; they print components of the source. If the field
 that you have selected is inapplicable to the source, latex will give an error. Most should be clear by their name; _\Prefix_ is the 
 leading number of a statute source, if applicable (such as the ``42'' in ``42 U.S.C. \S 1983''). _\SrcType_ is one of ``Case,'' ``Book,'' 
 ``Statute,'' or ``Other.'' _\SupraNote_ is the first footnote in which a source appears. _\LastNote_ is the most recent note
 in which a source appears.


 \cmd{SetIndexType}{Short Name}{New Index Type}[Change a source's destination index / table of authorities] 

 \noindent This command set the destination index for the source provided as their argument. The default index files are ``Case''
 ``Statute'' and ``Other,'' for cases, statutes, and everything else, respectively. _\SetIndexType_ can be used to alter the index file. 
 Note in particular that if _\SetIndexType_ is set to an empty string, indexing will be disabled for the given source. 
 The current index for a source can be queried with _\IndexType_. 


 \cmd{SetIndexName}{Short Name}{Name To Appear In Index}[Change the appearance of a source in the index / table of authorities]

 \noindent This can be used to provide additional detail in the Table of Authorities that would not be appropriate in the flow of the text.
 For example, consider the Federal Rules of Civil Procedure:

 _\newstatute{Fed. R. Civ. P.}{}_ \\
 \hspace*{\parindent}_\SetIndexName{Fed. R. Civ. P.}{Federal Rules of Civil Procedure !Rule }_

 \noindent The second line ensures that ``Federal Rules of Civil Procedure'' will be written long form in the Table, and that each rule will
 be a subentry under this heading, to be preceded by ``Rule.'' Currently, this has effect only for statutes.


~\par

\noindent
\begingroup\raggedright\hyphenpenalty=10000\hangindent=1.5em
\texttt{\bf \textbackslash{}See, \textbackslash{}Seealso, \textbackslash{}Seeeg, \textbackslash{}Seegenerally, \textbackslash{}Cf, \textbackslash{}Butsee, \textbackslash{}Butseeeg, \textbackslash{}Butcf, \textbackslash{}Compare, \textbackslash{}Contra, \textbackslash{}Accord} --- Introductory Signals 
\endgroup

 \begin{Example}
   \See \pincite{Marbury}{117}. \\
   \Seealso \pincite{Marbury}{120}. \\
   \Cf \pincite{Lochner}{602}. 
 \end{Example}

  \noindent These commands insert the italicized signal word in front of the citation,
  and cause the cite command to automatically handle correct capitalization of an ``\textit{id.},'' should one be used.
  Each command has both a capitalized (_\See_, _\Cf_, etc.) and a non-capitalized (_\see_, _\cf_, etc.) version.
  Also, in law review mode, they will correctly appear in the footnote with the citation, obviating the need to wrap 
  every source in a _\footnote_ command. 

\section{Miscellaneous}


 \noindent
_\S, \P_ --- Section and Paragraph symbols ``\S\unskip, \P\unskip,'' preserving following space.


 \noindent
_\ldots, \ldotss_ --- Non-breaking 3-dot and 4-dot Ellipses ``\ldots , \ldotss'' 

 \noindent
_\ellipsedotspacing_ --- Gap between dots in _\ldots_ (default \ellipsedotspacing) 


 \noindent
_\Ordinal, \ordinal_ --- Given an integer, convert into a textual ordinal number (through Tenth) 


 \noindent
_\availableat{URL}_ --- Provides a web citation in the text, using the URL package 


 \noindent
 _\makeandletter_ --- Allows ampersands to appear in the text.

 \noindent
 _\makeandtab_ --- Returns ampersands to their default definition.


\section{Configuration Options}

 \noindent _\CF, \BTF, \ATF, \BAF_ --- Fonts to be used in citations

 These define the fonts used for Cases, Book Titles, Article Titles, and Book Authors, respectively.
 In normal mode, the latter three are defined to _\em_, and _\BAF_ is empty (roman).
 In lawreview mode, cases are in roman font, Article Titles _\em_,  and Book Titles and Authors small-capped.

 These can be redefined as needed. Note that, because _\em_ is used, italics can be converted to underlines by including
 _\usepackage{ulem}_ at the start of the file.

~\par

\noindent {\bf\large Package Option _lawreview_} --- Switch to Law Review-type formatting 

 By adding this option to the package declaration, citations will display case titles in roman font, 
 while author and title are in small-caps. Citations will automatically be made into footnotes, by default 10pt. 
 If the source is combined with a introductory signal (\emph{See, Cf., etc.} ), the signal will also be 
 placed in the footnote. However, if a citation has multiple source in a string cite or parentheticals, 
 the entire footnote should be put into a LaTeX _\footnote_ command, or _\citetext_.
%
%\noindent Package Option _casesupra_ --- Use \emph{supra} in case short-form citations. 
%
% The Bluebook says that subsequent citations to a case should not use the word \textit{supra}
% to indicate that the case has been fully cited previously. However, in practice, this is 
% often seen, especially in Supreme Court opinions with their three parallel citations. 

\noindent _\indexglue, \indexpenalty_ --- Parameters for line-breaking Table of Authorities entries 

 In classes that create a table of authorities (such as _lawbrief.cls_), we encourage LaTeX to break
 long lines immediately after the Long Name, such that the citation is put on the next line
 (i.e. rather than putting a break in the middle of the citation). These two parameters control that,
 and generally will not need to be changed. The _\indexglue_ defaults to _0in plus 1.fil_, and _\indexpenalty_
 defaults to _-999_. 

\noindent _\maxsequentialids_ --- Maximum number of \textit{Id.}'s in a row

 After this number of sequential id's, we force a short cite for clarity. 
 Set it very large to disable this functionality.
 

\noindent _\forcelongevery_ --- Force a long citation after a source has not appeared in this many footnotes 

 The Bluebook has a ``5 Footnote Rule,'' in which a source that has not appeared in the last five footnotes is required
 to be provided as a long citation. Therefore, this parameter is set to 5 by default. Disable it by setting it to a large 
 value. Note that it has no effect unless in lawreview mode.
 

\section{Law Brief Document Class}
The file lawbrief.cls provides a flexible document class for appellate-style legal briefs,
including cover page, table of contents, tables of authorities, etc.
 \subsection{Sectioning}
 We redefine formatting for section commands and the associated numbering. By default,
 the top-level _\section_ (Question Presented, Statement of Facts, Argument, etc.) is unnumbered, bold and centered;
 _\subsection_ is bold and left-aligned with a roman numeral; _\subsubsection_ is bold, left-indented, and with a letter
 for ordering. (Note that the formatting in the table of contents differs somewhat, see below). By default, 
 _\hyphenpenalty=10000_ is set so as to disable any hyphenation of words in each heading.

 \subsection{Configuring the Index / Table of Authorities}
 

_\indexindentsize_ --- Hanging indentation of toa from the left side (default: \the\indexindentsize)


\noindent _\indexrightmargin_ --- Right margin for all-but-the-last lines of toa (default: \the\indexrightmargin)
 The current value will preserve .5 inch of space for non-final lines, and ragged margins. To change the spacing on
 the last line between the entry and the page numbers, you must edit the style file.

\noindent _\passimlimit_ --- Use ``passim'' above this many pages (default: \passimlimit).  
 Can be changed with \\ _\renewcommand{\passimlimit}{5}_, etc. 
 Make sure to set this before the index is created (i.e. in the header).


\noindent _\idxpassim_ --- Used in an index command, forces passim to always be used for a particular cite

\noindent
_\caseindextitle_ --- Set the title for the case section of the table of authorities (Default: \@caseindextitle).

\noindent
_\statutesindextitle_ --- Set the title for the statutes section of the table of authorities (Default: \@statutesindextitle).

\noindent
_\otherindextitle_ --- Set the title for the miscellaneous section of the table of authorities (Default: \@otherindextitle).


 \noindent _\tableofauthorities_ --- Print the table of authorities.


 \subsection{Table-of-Contents commands}


_\tableofcontents_ --- Print the table of contents.


\noindent _\@tocline_ --- Sets the leaders to print aligned dots. Redefine to change.


\noindent _\l@section_ --- Formats Top-level headings in the ToC flush left and small caps


\noindent _\l@subsection_ --- Formats Second-level headings indented from left with a hanging indent of _\@pnumwidth_


\noindent _\l@subsubsection_ --- Formats Third-level headings indented more from left with a hanging indent of _\@pnumwidth_


 \subsection{Page numbering commands}

 These commands are used by calling _\pagestyle{romanparen}_, _\thispagestyle{toa}_, etc.


 \noindent Pagestyle _arabicparen_ --- arabic numerals inside parenthesis at bottom


 \noindent Pagestyle _romanparen_ --- roman numerals inside parenthesis at bottom


 \noindent Pagestyle _toa_ --- ``inherits'' from romanparen, but also puts the current mark at the top


 \noindent Pagestyle _footertext_ --- Sets the argument to _\footertext_ in the footer, flush left

 \subsection{Title Page commands}

 \noindent _\firstparty_, _\secondparty_ --- Sets the name of the first and second parties in the title page or caption.

 \noindent _\firstpartytitle_, _\secondpartytitle_ --- Set the title (Plaintiff, Defendants, etc.) of the parties.

 \noindent _\plaintiff_, _\defendant_, _\appellant_, _\appellee_, _\petitioner_, _\respondant_ --- Set both the name and title of each party. 
 Note that each of these also has a plural version (e.g., _\plaintiffs_).

 \noindent _\titlegraphic_ --- Set the stylized header for the title page.

 \noindent _\maketitle_ 

 This creates the title page, and generally would be called immediately
 after the _\begin{document}_. Alternatively, you may call _\makefrontmatter_, which 
 will create the title, question presented (which per Supreme Court rules is on the page 
 immediately after the title), table of contents, and table of authorities.

 For additional changes, such as an overall formatting change, I suggest that 
 it would be easiest for you to modify the macro _\maketitle_ or _\makefrontmatter_ 
 in the class file directly.

 \noindent _\makecaption_ --- Creates a trial-brief style caption, as opposed to a cover page.


 \noindent _\makefrontmatter_

 Creates all the boilerplate front matter, including the title page (by calling _\maketitle_), 
 the question presented (taken from the argument of _\questionpresented_), the table of contents and 
 the table of authorities (by calling _\tableofcontents_ and _\tableofauthorities_, respectively).


 \noindent _\rightbox_ environment

 The contents of this environment will themselves be left-aligned, but the entire box 
 containing them will be as far to the right as possible. This is useful for signature-type
 boxes at the end of the brief.


\section{Arbitration Brief Document Class}
 The file _arbitrationbrief.cls_ provides a document class (derived from _memoir_) for arbitration-style briefs, such as is used
 in the annual Willem C. Vis International Arbitration Moot. It does not actually share much in the way
 of code with other LawTeX packages, but is included due to it's related subject matter.

 This class file was used for the Yale Law School respondant's submission to the 2012 Vis Moot. Much of the boilerplate text 
 (title page, headers, etc.) do not have commands provided to adapt them to other uses, therefore you will need to edit these 
 portions of the class file yourself.


\cmd{newauthority}{Type}{Short Name}{Full Name}[Declare a new authority]

 \noindent This is the _arbitrationbrief.cls_ equivalent of the LawTeX _\newcase_, etc. New authorities are declared with this command, to which one
 must provide an Authority Type (Commentary, Statutes, Rules, or Cases), a short name to appear in the body of the text, and a long name to 
 appear in the table of authorities.


\cmd{cite}{Short Name}[Cite an authority in a citation block]

 \noindent This adds a citation to the text, ensuring that it will be recorded in the table of authorities, and providing a hyperlink thereto.
 This command does not take a pin cite; that should be given explicitly in the flow of the text, if appropriate.


\cmd{romancite}{Short Name}[Cite an authority in the body text.]

 \noindent This adds a citation to the text, in roman (not italic) font, ensuring that it will be recorded in the table of authorities, and providing a hyperlink thereto.
 This command does not take a pin cite; that should be given explicitly in the flow of the text, if appropriate. 



\section{Final Notes}

Law\TeX{} is distributed for free (under the terms of the GPL) in the hope that the widest range of people may find it useful. Therefore if you are one of said people, I would really appreciate hearing from you. Additionally, I appreciate any bug reports, or ideas as to how to make Law\TeX{} more useful---even better if they come with proposed code! Finally, this software package was produced in an attempt at getting 99\% of the way to automated Bluebook citations, and with full recognition of the fact that 100\% is probably impossible. Therefore, there will always be some aspects of the Bluebook that this package will not cover. But I feel that the system I've put together is roughly at around that 99\%, so I do not anticipate making major changes in the future. 
 
\end{document}

