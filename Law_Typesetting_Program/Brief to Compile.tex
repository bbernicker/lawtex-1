\providecommand{\documentclassflag}{}
\documentclass[12pt,\documentclassflag]{SCOTUS_Brief}
\MakeOuterQuote{"}

%Paste Citations Here
\newstatute{Mo. Ann. Stat.}{(2021)}
\citecase[Casey]{Planned Parenthood of Southeastern Pennsylvania v. Casey, 505 U.S. 833 (1992)}
\citecase[Roe]{Roe v. Wade, 410 U.S. 113 (1973)}
\citecase[Carhart]{Stenberg v. Carhart, 530 U.S. 914 (2000)}
\citecase[Lawrence]{Lawrence v. Texas, 539 U.S. 558 (2003)}
\citecase[Box]{Box v. Planned Parenthood of Indiana and Kentucky, Inc., 139 S.Ct. 1780 (2019)}
\newstatute{18 Pa.C.S.A.}{(2021)}
\newstatute{N.D. Cent. Code Ann.}{(2021)}
\newstatute{La. Stat. Ann.}{(2021)}
\citecase[Isaacson]{Isaacson v. Brnovich, No. CV-21-01417-PHX-DLR, 2021 WL 4439443 (D. Ariz. Sept. 28, 2021)}
\citecase[Hopkins]{Hopkins v. Jegley, 510 F. Supp. 3d 638 (E.D. Ark. 2021)}
\citecase[Memphis]{Memphis Ctr. for Reprod. Health v. Slatery, 14 F.4th 409 (6th Cir. 2021)}
\newstatute{Ariz. Rev. Stat. Ann.}{(2021)}
\newstatute{Ark. Code}{(2021)}
\newstatute{Kan. Stat. Ann.}{(2021)}
\newstatute{N.C. Gen. Stat.}{(2021)}
\citecase[Roberts]{Roberts v. U.S. Jaycees, 468 U.S. 609, 610 (1984)}
\newmisc{Respondents' Brief in Casey}{Brief of Respondents in \textit{Planned Parenthood of Southeastern Pa.} v. \textit{Casey}, O. T. 1991, Nos. 91–744, 91–902, p. 4, 1992 WL 12006423\pin{, }{}}
\newbook{Sheffer}{Edith Sheffer}{Asperger's children: The origins of autism in Nazi Vienna}{(2018)}
\citecase[Marks]{Marks v. United States, 430 U.S. 188 (1977)}

%Set the information for the title page (later produced by \makefrontmatter)
\docket{No. 21-3}
\circuit{Eighth}
\petitioner{Eric S. Schmitt, Attorney General of Missouri, et al.}
\respondent{Reproductive Health Services of Planned Parenthood \\ of the St. Louis Region, Inc.}
\briefposture{On Writ of Certiorari to\\ The United States Court of Appeals\\ for the Eighth Circuit}
\nameofbrief{Brief for the Petitioners}
\author{Brendan Bernicker}
\authorrole{Counsel for the Petitioners}
\authorfirm{Yale Law School \\ Con. Lit. Seminar}
\authoraddress{127 Wall Street \\ New Haven, CT 06510}
\authorphone{(610) 203-0293}
\authoremail{brendan.bernicker@yale.edu}
\filingdate{27th day of October, 2021} %format is important for certificate of service
\pythonwordcount{PYTHON WILL INPUT THE WORD COUNT HERE}

\disclosurestatementrequired{0} %Yes(1)/No(0), tells the computer whether to create page for it
\disclosurestatement{{If required, the disclosure statement goes here.}}

\questionpresented{{Whether a Missouri Law that prohibits medical providers from performing abortions when the provider knows that the sole reason for the abortion is a screening or test that indicates the unborn child has Down's Syndrome (the Down's Syndrome provision) is facially invalid under \textit{Roe v. Wade} and \textit{Planned Parenthood of Southeastern Pennsylvania v. Casey} or whether it is a valid, reasonable regulation of abortion.}}

\begin{document}
%This commands creates the title page, table of contents, and table of authorities
\makefrontmatter

%Sets the formatting for the entire document after the front matter
\parindent=2em
\setlength{\parskip}{1.25ex plus 2ex minus .5ex}
\setstretch{2}  %this is double spacing

\section{Opinions Below}

The opinion of the Court of Appeals is reported at 1 F.4\textsuperscript{th} 552 (8\textsuperscript{th} Cir. 2021). The opinions of the District Court are reported at 389 F. Sup. 3d 631 (W.D. Mo. 2019) and 408 F. Sup. 3d 1049 (W.D. Mo. 2019).

\section{Jurisdiction}

The judgment of the court of appeals was entered on June 9, 2021. The petition for a writ of certiorari was filed on June 30, 2021, and was granted on September 1, 2021. This Court’s jurisdiction rests on 28 U.S.C. § 1254(1).

\section{Statutory Provisions Involved}

The statutory provision at issue in this case is set out in Mo. Ann. Stat. § 188.038. It reads, in relevant part, "[n]o person shall perform or induce an abortion on a woman if the person knows that the woman is seeking the abortion solely because of a prenatal diagnosis, test, or screening indicating Down Syndrome or the potential of Down Syndrome in an unborn child." The full section of the Missouri Code is attached as Appendix A.

\section{Introduction}

\textit{Roe v. Wade} searched the "penumbras" of the Bill of Rights and found a substantive due process right to terminate a pregnancy. \pincite{Roe}{152}. It then held that this right is "fundamental," and that laws which burden it are therefore subject to strict scrutiny. \pincite{Roe}{155}. There are good reasons to question both of these holdings. \Seeeg \pincite{Carhart}{980-1}~(\textsc{Thomas}, J., dissenting).

But this Court need not reconsider \textit{Roe} or \textit{Casey} \cite[!]{Casey} to uphold Missouri's prohibition on terminating a pregnancy solely because the unborn child has Down Syndrome. \textit{Roe} \cite[!]{Roe} and \textit{Casey} \cite[!]{Casey} weighed a woman's right to terminate a pregnancy against two state interests: (1) protecting the health of pregnant women and (2) protecting the 'potential life' of unborn children. \textit{Roe's} "essential holding," which \textit{Casey} clarified, was that neither interest is compelling enough to support categorical bans on pre-viability abortions. But \textit{Casey} explicitly left open the possibility that "other interests \ldots  [may] justify the regulation of abortion before viability" in certain circumstances. \pincite{Casey}{914}~(Stevens, J., concurring in part and dissenting in part); \seealso \cite{Marks}.

More than a dozen states have determined that preventing invidious discrimination against certain protected classes is a compelling interest that can support bans on abortions sought solely for certain reasons. These so-called "reason bans" build on \textit{Roe's} \cite[!]{Roe} recognition that a woman does not have a right to have a pre-viability abortion "for whatever reason she alone chooses." \pincite{Roe}{153}. As this Court recently noted, it has not yet decided whether these "sex-, race-, and disability-selective abortions" are constitutional. \pincite{Box}{1792}.

They are. \textit{Roe} \cite[!]{Roe}, \textit{Casey} \cite[!]{Casey}, and their progeny held that abortion bans are subject to strict scrutiny. Bans that are narrowly tailored to advance a compelling state interest are constitutional, even if they substantially burden the right to terminate a pregnancy. 

Missouri has a compelling interest in eradicating invidious discrimination against people with Down's Syndrome. This interest is particularly important in the light of our country's history of marginalizing, institutionalizing, and even sterilizing people with Down's Syndrome. The Missouri legislature considered this history and contemporary evidence and determined that aborting "unborn children with Down Syndrome raises grave concerns for the lives of those who do live with disabilities. It sends a message of dwindling support for their unique challenges, fosters a false sense that disability is something that could have been avoidable, and is likely to increase the stigma associated with disability." \pincite{Mo. Ann. Stat.}{ \S 188.038}.

The Missouri legislature reasonably concluded that each abortion based solely on the unborn child's Down's Syndrome (or potential Down's Syndrome) was an instance of invidious discrimination that the state had a compelling interest in eradicating. It therefore adopted a narrowly tailored statute that only reaches abortions motivated by discriminatory intent. Missouri's statute does not prohibit all abortions, all abortions of unborn children with Down's Syndrome, or even all abortions based solely on a Down's Syndrome diagnosis.\footnote{There may be circumstances where a woman is actually seeking an abortion solely because of a Down's Syndrome diagnosis, but where her doctor is not aware of her motivation (or only suspects or believes, rather than knows, that the diagnosis is her sole motivation).} Instead, it prohibits doctors from performing abortions when they know, from the patient's statements or from other circumstantial evidence, that the unborn child's Down's Syndrome is the sole reason for the abortion.

Because Missouri's ban on abortions sought solely because of a Down Syndrome diagnosis is narrowly tailored to advance the state's compelling interest in eradicating discrimination against people with Down Syndrome, it is constitutional even under \textit{Roe} \cite[!]{Roe}. But if the Court holds that Missouri's reason ban cannot withstand strict scrutiny, it should nonetheless uphold the ban. In our federal system, courts must respect a state's duly enacted legislation and may only strike down state laws that violate the Constitution.

The Constitution does not mention abortion. Nothing in the Constitution's text, history, or structure removes the question of when and whether abortion should be permitted from the democratic process. Yet when the people's democratically-elected representatives enact legislation that burdens abortion, courts subject that legislation to a higher level of scrutiny than statutes which burden most specifically-enumerated constitutional rights. Even assuming that it is proper to read a right to terminate a pregnancy into the Due Process clause, it does not follow that laws which burden that right must be subjected to strict scrutiny. Missouri has a rational basis for its reason ban and, under a proper construction of the Fourteenth Amendment, that is enough for the ban to be constitutional.

\section{Statement of the Case}

The Court of Appeals held that pre-viability bans on discriminatory abortions are \textit{per se} unconstitutional under this Court's precedents. This is not so.

\section{Summary of the Argument}

This Court's precedents permit bans on certain categories of pre-viability abortions so long as those bans are narrowly tailored to advance a compelling government interest. Missouri's law clears this high bar.

The Missouri statute at issue in this case sets out three reason bans. It provides, "No person shall perform or induce an abortion on a woman if the person knows that the woman is seeking the abortion solely because of [1] the sex or [2] race of the unborn child [or] \ldots  [3] prenatal diagnosis, test, or screening indicating Down Syndrome or the potential of Down Syndrome in an unborn child." 

If this Court disagrees, however, that Missouri's statute withstands strict scrutiny, it cannot strike down the anti-discrimination provision without first considering whether prohibitions of this type ought to be subjected to strict scrutiny at all. In our federal system, Courts must respect a co-equal sovereign's duly enacted legislation unless the Court is convinced that the legislation violates the U.S. Constitution. The Constitution does not mention abortion, and nothing in its text, history, or structure suggests that abortion is so fundamental a right that it deserves a level of scrutiny that even some enumerated rights do not receive. Even assuming that it is proper to read a right to terminate a pregnancy into the Due Process clause, it does not follow that laws which burden that right must be subjected to strict scrutiny. In is undisputable that Missouri has a rational basis for its anti-discrimination provision and, under a proper construction of the Fourteenth Amendment, that is enough for the Court to uphold Missouri's law.

\textit{Roe} and \textit{Casey} \cite[!]{Casey} require courts to subject reason bans to strict scrutiny but, where reason bans are narrowly tailored to advance the compelling state interest in eradicating invidious discrimination, are constitutional even when they substantially burden the right to terminate a pregnancy.

The Court of Appeals focused on a single statement, offered in dicta in a plurality opinion in \textit{Casey}, which says "a State may not prohibit any woman from making the ultimate decision to terminate her pregnancy before viability." \pincite{Casey}{879}~(O'Connor, Kennedy, Souter, JJ., joint opinion). Justice Stevens's opinion, however, specifically left open the possibility that interests other that the state's "legitimate interest in potential human life \ldots  [may] justify the regulation of abortion before viability." \pincite{Casey}{914}~(\textsc{Stevens}, J., concurring in part and dissenting in part). Because there was no opinion of the Court as to this question, Justice Stevens's narrower understanding of \textit{Casey's} holding controls. \Seegenerally \cite{Marks}.

"The meaning of any legal standard can only be understood by reviewing the actual cases in which it is applied." \pincite{Casey}{920, n. 6}~(\textsc{Stevens}, J., concurring in part and dissenting in part).

\textit{Roe} \cite[!]{Roe} and \textit{Casey} \cite[!]{Casey} weighed a woman's right to terminate a pregnancy against two state interests: (1) protecting the health of pregnant mothers and (2) protecting the 'potential life' of unborn children. The Court held in \textit{Roe}, and reaffirmed in \textit{Casey}, that neither interest was compelling enough for abortion bans to satisfy strict scrutiny in the early stages of pregnancy. \textit{Roe} \cite[!]{Roe} and \textit{Casey} \cite[!]{Casey} left open the possibility, however, that other state interests might provide a compelling basis for regulating pre-viability abortion in certain circumstances.

Reason bans have existed since before \textit{Casey} \cite[1]{Casey}. In fact, the statute at issue in \textit{Casey} banned sex-discriminatory abortions, but the \textit{Casey} plaintiffs did not challenge that reason ban and it remains in effect. \See \pincite{18 Pa.C.S.A.}{\S 3204(c)}; \seealso \pincite{Respondents' Brief in Casey}{4}~(noting that there was no challenge to the sex-selective abortion bar). Similarly, the respondents here only contest one of the three reason bans in the Missouri statute. 

For example, this Court has recognized that states have "a compelling interest in eradicating discrimination against [their] female citizens." \pincite{Roberts}{610}. To vindicate this interest, nine states, including Missouri, have prohibited women from terminating a pregnancy solely because of the sex of the fetus.\footnote{Copy here and check every statute again.} A

These prohibitions are narrowly tailored to advance the state's interest.\footnote{Some of the state statutes are not limited only to abortions motivated "solely" by the sex of the fetus. \Seeeg North Carolina (Get the parenthetical), Tennessee.} They do not reach all abortions, or even all abortions of female fetuses. They instead focus on discriminatory intent and only prohibit abortions that are motivated solely by the sex of the unborn child. These prohibitions therefore withstand strict scrutiny. Indeed, the respondents in this case do not challenge Missouri's prohibition on sex discrimination in abortion decisions, just as the plaintiff's in \textit{Casey} \cite[!]{Casey} did not challenge Pennsylvania's prohibition on sex-selective abortion when they challenged other provisions of the same statute. \See \pincite{18 Pa.C.S.A.}{\S 3204(c)}; \seealso \pincite{Respondents' Brief in Casey}{4}~(noting that there was no challenge to the sex-selective abortion bar).

The same reasoning that supports the unchallenged prohibitions on sex-discriminatory abortions also supports Missouri's ban on abortions that are sought solely because of a Down Syndrome diagnosis. First, Missouri has a compelling state interest in eradicating discrimination against people with Down Syndrome. The long and shameful history of government-supported measures to institutionalize and sterilize people with Down Syndrome only strengthens Missouri's interest in eradicating this discrimination.

For an abortion restriction aimed at this

The statutory provision in this case is not designed to protect the health of pregnant women or to safeguard the potential life of unborn children. Instead, it advances the state's compelling interest in eradicating discrimination against people with Down Syndrome.

Specifically, the Court has never considered whether the state's compelling interest in preventing discrimination against historically oppressed and marginalized groups can justify prohibitions on "sex-, race-, and disability-selective abortions." \pincite{Box}{1782}.

As set forth below, in the decades immediately following \textit{Roe}, even abortion advocates generally accepted that states could prevent discriminatory abortions.

The Missouri law at issue in this case is narrowly tailored because it only burdens women who seek an abortion \textit{solely} because their unborn child has Down Syndrome.

\section{Argument}

\subsection{States can ban categories of pre-viability abortions as long as the bans are narrowly tailored to advance a compelling state interest.}

In \textit{Casey}, Justice Stevens explicitly acknowledged that while states' "legitimate interest in potential human life \ldots  does not justify the regulation of abortion before viability \ldots  other interests, such as maternal health, may." \pincite{Casey}{914}~(\textsc{Stevens}, J., concurring in part and dissenting in part). Because there was no opinion of the Court as to this question in \textit{Casey}, Justice Stevens's narrower understanding of \textit{Casey's} holding controls. \Seegenerally \cite{Marks}.

\subsection{This Court has never recognized a right to discriminatory abortions.}

This Court has never considered whether the state's compelling interest in preventing discrimination against historically oppressed and marginalized group can justify prohibitions on discriminatory abortions based on protected characteristics. \pincite{Box}{1782}~(noting that the Court has not yet considered prohibitions on "sex-, race-, and disability-selective abortions.").

\textit{Roe} \cite[!]{Roe} and \textit{Casey} \cite[!]{Casey} considered states’ interests in enacting general abortion regulations and held that they were not compelling with respect to pre-viability abortions. Neither case, however, held that states could not have a compelling interest in regulating specific types of abortions. Instead, this Court’s decisions in other contexts have made plain that states do have compelling interests in eradicating discrimination, and those interests are no less compelling in the abortion context than they are in other contexts.

[Restate Casey’s “essential holding” paraphrase of \textit{Roe}].

The state interests in general abortion regulations recognized in \textit{Roe} and \textit{Casey} are not exhaustive.

“Private biases may be outside the reach of the law, but the law cannot, directly or indirectly, give them effect.” Palmore v. Sidoti, 466 U.S. 429, 433, 104 S.Ct. 1879, 1882, 80 L.Ed.2d 421 (1984).

\subsection{Missouri has a compelling interest in preventing discriminatory abortions based solely on Down Syndrome diagnoses.}

While \textit{Roe} \cite[!]{Roe} and \textit{Casey} \cite[!]{Casey} do not require women to offer any reason for seeking an abortion, it does not follow that a woman may have an abortion "for whatever reason she alone chooses." \pincite{Roe}{153}. There are many areas of law where somebody is free to take an action for any reason except based on a protected characteristic. An employer can lay off an employee for no reason, but cannot lay off an employee because she becomes pregnant. A restaurant owner can serve or refuse to serve whoever they like, but they cannot refuse to serve people on account of their race. Similarly, a woman can choose to terminate her pregnancy prior to viability for no reason at all or for almost any reason; but, in Missouri, she cannot choose to abort an unborn child solely because of that child’s sex, race, or Down Syndrome diagnosis.

In the decades immediately following \textit{Roe}, even abortion advocates generally accepted that states could prevent discriminatory abortions. For example, the Pennsylvania statute at issue in \textit{Casey} prohibited sex-selective abortions, but Planned Parenthood of Southeastern Pennsylvania opted not to challenge that provision and it remains in effect to this day. \See \pincite{18 Pa.C.S.A.}{\S 3204(c)}; \seealso \pincite{Respondents' Brief in Casey}{4}.

Respondents do not challenge the provisions of Missouri’s statute that prohibit abortions based solely on a fetus’s sex or race. Including Missouri and Pennsylvania, ten states ban some or all sex-discriminatory pre-viability abortions. \footnote{\See \pincite{Ariz. Rev. Stat. Ann.}{ \S 13–3603.02}~(the portions of this statute pertaining to discrimination based on Down syndrome diagnoses have been enjoined in \cite{Isaacson}, but the sex-discrimination provision was not challenged and remains in effect); \pincite{Ark. Code}{ \S 20–16–1904}~(other portions of this statute have been enjoined in \cite{Hopkins}, but the sex-discrimination provision was not challenged and remains in effect); \pincite{Kan. Stat. Ann.}{ \S 65–6726}; \pincite{N.C. Gen. Stat.}{ \S 90–21.121}; \pincite{N.D. Cent. Code Ann.}{ \S 14–02.1–04.1}} Most of these sex-discrimination bans have never been challenged.\footnote{Federal courts have enjoined the enforcement of statutes banning sex-discriminatory pre-viability abortions in four additional states for reasons that are inapplicable to the Missouri statutes at issue in this case. The Sixth Circuit held that a Tennessee statute prohibiting abortions based on sex, race, or a Down Syndrome diagnosis was unconstitutionally vague because, unlike the Missouri statute, it did not limit liability to situations where the discriminatory basis was the \textit{sole} reason for the abortion. \Seegenerally \cite{Memphis}. Illinois, Indiana, and Kentucky are the others.}

These sex-discrimination statutes do not serve either of the state interests addressed in \textit{Roe} or \textit{Casey}, and nobody disputes that they substantially burden abortions based on discriminatory factors. But their often-unchallenged persistence and adoption demonstrate that states have compelling interests in rooting out discrimination, and that outright bans on certain discriminatory abortions can survive strict scrutiny even when they pose a substantial burden.

North Dakota has prohibited abortions based solely on a Down Syndrome diagnosis since 2013. \pincite{N.D. Cent. Code Ann.}{\S 14-02.1-04.1} ~(West)~(also prohibiting sex-selective abortions). Louisiana has a similar law that prohibits pre-viability abortions after 20 weeks if those abortions are solely motivated by a diagnosis of a genetic abnormality. \pincite{La. Stat. Ann.}{\S 40:1061.1.2}. Neither law has been challenged.

"Thirty per cent of those who were killed suffered from physical disabilities, one in ten had Down Syndrome, others suffered from hydrocephalus, epilepsy, cerebral palsy and other brain injuries or disorders. Socially related 'coexisting indications' in the parents, such as alcoholism, speech defects or sexual impulsiveness, were often added to the applications for permission to kill the children." \pincite{Sheffer}{185} [NOT A DIRECT QUOTE. FROM https://tidsskriftet.no/en/2019/05/essay/asperger-nazis-and-children-history-birth-diagnosis]

\textit{Roe} also explicitly rejected the view that a pregnant woman has the right "to terminate her pregnancy at whatever time, in whatever way, and for whatever reason she alone chooses." \pincite{Roe}{153}. The \textit{Roe} Court approvingly cited lower court opinions holding that neither of the two state interests the Court recognized "justified \textit{broad} limitations on the reasons for which a physician and his pregnant patient might decide that she should have an abortion in the early stages of pregnancy." \pincite{Roe}{156}~(emphasis added). Where states have compelling interests other than the two \textit{Roe} rejected, \textit{Roe} does not prohibit them from adopting specific, narrowly-tailored limitations on the reasons for which a woman can decide to terminate her pregnancy.

Past discrimination in the form of state eugenics program. (Did Missouri have one?)

Statement that Missouri values people with down syndrome would be empty if Missouri allowed women to decide that it would be per se better that people with Down Syndrome never be born.

Aborting an unborn child solely because they have down syndrome is akin to expressive hate speech.

Relatedly, Missouri has an interest in maintaining neurological diversity. [This statute is about diversity and dignity.]

\subsection{Missouri's prohibition on discriminatory abortions is narrowly tailored.}

As the 6,000 mothers who give birth to children with Down Syndrome each year demonstrate, nobody has an abortion \textit{solely} because of a Down Syndrome diagnosis. Some women may choose an abortion because their unborn child has Down Syndrome \textit{and} they are not emotionally prepared to carry any child to term regardless of their special needs. That abortion would not be solely due to the diagnosis. Some women may chose to have an abortion because their unborn child has Down Syndrome \textit{and} they are unable to raise a child with Down Syndrome but unwilling to put their own child up for adoption. That abortion would not be solely due to the diagnosis.

Indeed, basic rules of textualism and lenity will make it hard for Missouri to ever bring a prosecution under this statute.

A criminal statute can stigmatize the conduct it prohibits, even when that statute is very rarely enforced. ("If protected conduct is made criminal and the law which does so remains unexamined for its substantive validity, its stigma might remain even if it were not enforceable as drawn for equal protection reasons. When homosexual conduct is made criminal by the law of the State, that declaration in and of itself is an invitation to subject homosexual persons to discrimination both in the public and in the private spheres.") \pincite{Lawrence}{575}. We want to stigmatize this conduct.

\subsection{If the Court holds that this statute does not pass strict scrutiny, it should reconsider whether the right recognized in \textit{Roe} is fundamental.}

Federalism is fundamental. I'll tolerate your sins and and the laboratories of democracy.

Democratic self-government is fundamental. Case for originalism.

"The States may, if they wish, permit abortion on demand, but the Constitution does not \textit{require} them to do so. The permissibility of abortion, and the limitations upon it, are to be resolved like most important questions in our democracy: by citizens trying to persuade one another and then voting." \pincite{Casey}{979}~(emphasis original)~(\textsc{Scalia}, J., concurring in the judgment in part and dissenting in part).

\section{Conclusion}

For the foregoing reasons, the judgment of the Court of Appeals should be reversed.


\makeendmatter 

\end{document}
