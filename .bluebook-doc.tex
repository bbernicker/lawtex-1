\parskip=6pt
\parindent=1.5em
\leftskip=0em
\frenchspacing
\font\parasymbolfont=phvr at 10pt
\font\sectionsymbolfont=pncr at 10pt
 %\RequirePackage{xr} % in preamble
 \section{Defining Sources}\vspace{-8pt}

 \citecase[Steel Seizure Case]{Youngstown Sheet & Tube Co. v. Sawyer, 343 U.S. 579 (1952)}
 \citecase[Menard]{Menard, Inc. v Escanaba, 315 Mich. App. 512; 891 N.W.2d 1 (2016)}
\cmd{citecase}[Short Name]{Standard Case Citation}[Defines a new case citation, using standard citation form.]
 \begin{Example}
   %\citecase{Lochner v. New York, 198 U.S. 45 (1905)}
   %\citecase[Steel Seizure Case]{Youngstown Sheet & Tube Co. v. Sawyer, 343 U.S. 579 (1952)}
   \cite[l]{Lochner}. \\ 
   \pincite{Lochner}{48}. \\
   \pincite{Steel Seizure Case}{602}. \\
   \pincite{Lochner}{52}. \\
   
   %\citecase[Menard]{Menard, Inc. v Escanaba, 315 Mich. App. 512; 891 N.W.2d 1 (2016)} \\
   \pincite{Menard}{513; 2} \\ % will cite both reporters. 
   \pincite{Menard}{\_; 2} \\ % will cite both reporters for long citations, but only the second reporter for short cites. The single \_ indicates that the writer does not know where the quote is in the first reporter. If this is a short cite only, the first reporter will be skipped and only the second reporter will appear. If you want to force appearance of the first reporter, just use something different than a single \_, a \_\_ will work. This hack accommodates situations where the official reporter is not available, such as when only a copy of the unafficial reporter is available online.
 \end{Example}
 \noindent In order to cite legal sources in the body of the text, first include a _\citecase_ command with the 
 case citation to define the case as a source. At the point in the text where the citation should appear, use
 _\cite_ or _\pincite_, passing the short name as an argument.  Spacing and commas are important (extra spaces will be preserved). 

 By default, the first party is used for the short name of the case, unless that party is 
 ``United States,'' ``State,'' ``Commonwealth,'' or ``People,'' in which case the second party is used.  
 To override the choice of short name, set the optional parameter. 

 With case citations that differ from the standard format, you may need to use the _\newcase_ command.
 For example, in ``\emph{Marbury v. Madison}, 5 U.S. (1 Cranch), 137 (1803)'', the extra parentheses around
 the ``(1~Cranch.)'' will throw off the parser. See the way that this is cited below. 

 \cmd{addReference}{[AppendixName]}{ShortName}{reference name}[Links a case to an appendix record. To use this with an external appendix, put in the preamble: \\RequirePackage{xr} and \\externaldocument{appendix-file-name}.]
 \citecase[Patru]{Patru v City of Wayne, unpublished per curiam opinion of the Court of Appeals, issued May 8, 2018 (Docket No. 337547)}
 \addReference[sample-appendix]{Patru}{patruvwayne}
 \begin{Example}
   %\RequirePackage{xr} % in preamble
   %\externaldocument{sample-appendix} % in preamble
   %\citecase[Patru]{Patru v City of Wayne, unpublished per curiam opinion of the Court of Appeals, issued May 8, 2018 (Docket No. 337547)}
   %\addReference[Sample Appendix]{Patru}{patruvwayne}
   \pincite{Patru}{3}
 \end{Example}
 \noindent The _\caseReference_ command links a case to an appendix. Call it after you've defined a case. Whenever you cite to the case, a reference will be made to the right page in the Appendix. In Michigan, unpublished cases must be provided. This command makes it easy for people to look up the case in the appendix.

 
\cmd{newcase}{Short Name}{Full Name}{Reporter}{Starting Page}{Parenthetical}[Defines a new case citation.]
 \newcase{Marbury}{Marbury v. Madison}{5 U.S. (1 Cranch)}{137}{(1803)}
 \begin{Example}
   %\newcase{Marbury}{Marbury v. Madison}{5 U.S. (1 Cranch)}{137}{(1803)}
   \pincite{Marbury}{140}.
 \end{Example}

 \noindent The _\newcase_ command is equivalent to the _\citecase_ command (which in fact calls _\newcase_ internally). 
 Generally, it is simpler to use the _\citecase_ command, but _\newcase_ is necessary for case citations that differ from the usual, 
 such as \textit{Marbury}, above, because of the parentheses around the ``(1 Cranch)''. 

% Actually invoke the commands for the documentation:


\cmd{newbook}{Short Name}{Authors}{Title}{Parenthetical}[Define a new book citation.]
 \begin{Example}
   \newbook{Prosser and Keaton}{William Lloyd Prosser & W. Page Keaton}
     {The Law of Torts}{(2nd ed., 1953)} 
   \newbook{Schelling}{Thomas Schelling}
     {A Process of Residential Segregation: Neighborhood Tipping, 
     {\upshape\it reprinted in} Economic Foundations of Property Law {\upshape 307,}}
     {(Bruce A. Ackerman ed., 1975)}
   \pincite{Schelling}{308} \\
   \pincite{Prosser and Keaton}{vol. 2, 15}. \\
   \pincite{Prosser and Keaton}{vol. 2, 345}. \\
   \pincite{Prosser and Keaton}{vol. 4, 876}. \\ 
   \pincite{Schelling}{310}.
 \end{Example}
 
 \noindent This command introduces a new book citation. Such citations are first written with a 
 long name that includes [volume] author, title, citation and parenthetical. Subsequent invocations will use the short name given by the first argument,
 which is also used as the argument to _\cite_, followed by \textit{supra}. In law-review mode, this is further followed by ``note n,'' the first 
 footnote in which the article was cited. Note that any _\textit_ or _\emph_ and their braces are stripped out of the short name (see _\newarticle_ for an example). 

 To cite a particular volume, use a pincite in the form _\pincite_\arg{Short Name}\{vol. 1, 123\}. That is, it must start with ``vol.''
 followed by exactly one space, then the number followed by a comma and another space. 

 This command is also used for a \textit{reprinted in} citation. Note the use of _\upshape_ and _\it_ (or equivalent) that is necessary
 to ensure the correct formatting.


\cmd{newarticle}{Short Name}{Authors}{Title}{Journal}{Start Page}{Parenthetical}[Define a new article citation.]
 \begin{Example}
   \newarticle{Note, \textit{The Ministerial Exception}}
     {Note}{The Ministerial Exception To Title VII}
     {121 Harv. L. Rev.}{1776}{(2009)}
   \newarticle{Ward}{Barbara Ward}{Progress for a Small Planet}
     {Harv. Bus. Rev.}{Sept.--Oct. 1979, at 89}{}
   \cite[l]{Note, The Ministerial Exception}.  \\
   \pincite{Ward}{90}. \\
   \pincite{Note, The Ministerial Exception}{1800}. 
 \end{Example}

 \noindent This command introduces a new law-review article-type citation. Such citations are first written with a 
 long name that includes author, title, citation and parenthetical. Subsequent invocations will use the short name given by the first argument,
 which is also used as the argument to _\cite_, followed by \textit{supra}. In law-review mode, this is further followed by ``note n,'' the first 
 footnote in which the article was cited. Note that any _\textit_ or _\emph_ and their braces are stripped out of the short name. 

 Note that in law review mode, the journal name is by default set in \textsc{Small Caps}, which is the Bluebook standard for 
 consecutivly-paginated journals. To produce standard type, use the
 form of _{{\upshape Harv. L. Rev.}}_ when defining the citation. 


\cmd{newincollection}{Short Name}{Authors}{Article Title}{Collection Title}{Page}{Parenthetical}[Defines a new article/chapter-in-collection citation.]

 \begin{Example}
   \newincollection{Allen, \textit{Oration}}{John Allen}
     {Oration Upon The Beauties Of Liberty}
     {Political Sermons of the American Founding Era}{vol. 1, 58}
     {(Ellis Sandoz ed., 1991)}
   \newincollection{Mather}{Moses Mather}{America's Appeal To The Impartial World}
     {Political Sermons of the American Founding Era}{vol. 1, 103}
     {(Ellis Sandoz ed., 1991)}

   \pincite{Allen, Oration}{62}. \\
   \pincite{Mather}{103}. \\
   \pincite{Allen, Oration}{78}. \\
   \pincite{Mather}{119}. \\
 \end{Example}

 Note that the parenthetical will be placed after the collection title, and therefore not printed if the collection itself is cited a second time. Note additionally
 that if the parenthetical changes across definitions for multiple documents in the collection, whichever source is actually cited first defines the parenthetical that is used.
 

\cmd{newinsingleauthorcollection}{Short Name}{Author}{Title}{Collection Title}{Page}{Parenthetical}[Defines a citation to a single-author collection.]

 \begin{Example}
   \newinsingleauthorcollection{Holmes}{Oliver Wendell Holmes}
     {Law in Science and Science in Law}{Collected Legal Papers}
     {210}{(1920)}
   \pincite{Holmes}{vol. 1, 120}. \\
   \pincite[s]{Holmes}{vol. 1, 133}. 
 \end{Example}

 \noindent This command is similar to _\newincollection_, for collections whose documents are all from the same author. The Bluebook specifies that these should
 be cited like a book, and therefore with the volume number before the author, and the author's name in small caps.


\cmd{newstatute}[optional handle@index-place]{Short Name}{Parenthetical}[Define a new statute citation.]
 \begin{Example}
   \newstatute{42 U.S.C.}{(2006)}
   \newstatute{Administrative Procedure Act}{(2006)}
   \pincite{Administrative Procedure Act}{\S 1, 5 U.S.C. \S 551}. \\
   \pincite{Administrative Procedure Act}{\S 2}. \\
   \pincite{42 U.S.C.}{\S 1983}.
 \end{Example}


 Add position indexing and separate shortnaming.

1) The references in the Statutes index can appear in the wrong order or
 2) can involve formatting like small caps (\textsc) which cannot be used as
 the internal TeX handle. This change solves both problems. A new
 optional argument is added to \newstatute. The argument can take the
 following forms:

 \newstatute[AA@]{MCL}{} - put the MCL statute at place AA in the index.

 \newstatute[am1]{\textsc{U.S. Const.} amend. I} - use am1 as the handle
 for the first ammendment. (The \textsc prevents the name from being used
 as a handle.)

 \newstatute[1@amendment1]{\textsc{U.S. Const.} amend. I} - use amendment1 as the
 handle for the first ammendment, put it at position 1 in the index.

\cmd{newmisc}{Short Name}{Full Name}[Define a general source by explicitly providing long and short citations.]
 \begin{Example}
   \newmisc{Bill of Rights 1689}{Act Declaring the Rights and Liberties of the 
     Subject and Settling the Succession of the Crown (Bill of Rights), 
     1 W. & M., sess. 2 c. 2\pin{, }{} (1689)}
   
   \pincite{Bill of Rights 1689}{\S 2}. \\
   \pincite[s]{Bill of Rights 1689}{\S 3}.
 \end{Example}

 \noindent The long cite may have the command _\Pin_\arg{Text Before}\arg{Text After}, which in the case of 
 a pincite, will insert the cite at the given location, surrounded by the text as indicated.
\section{Citing Sources}

\noindent\texttt{\large\bf\textbackslash{}[pin]cite} --- Cite a legal source using Bluebook style

\noindent Usage: 

 \texttt{\textbackslash{}cite}[\textit{Formatting}]\{\textit{Short Name}\} \\
 \hspace*{\parindent}\texttt{\textbackslash{}pincite}[\textit{Formatting}]\{\textit{Short Name}\}\{\textit{Pin Page}\} 

 \begin{Example}
   \cite[l]{Marbury}. \\ 
   \pincite[i]{Marbury}{117}. \\
   \pincite[s]{Marbury}{117}. \\
   \pincite{Steel Seizure Case}{602}. \\
   \pincite{Marbury}{122}. \\
   \See \pincite[n]{Steel Seizure Case}{625}. \\
   \Cf \pincite[s]{Steel Seizure Case}{625}.
 \end{Example}
	 
 \noindent The optional first argument forces a particular citation form, which is useful where the correct form cannot be determined automatically
 (\textit{e.g.}, the rule that one may not use \Id. in the next citation after a string cite), or that one does not capitalize id. 
 when it appears in the middle of a sentence (although when using the signal macros _\See_, _\Seealso_, etc.) this will be
 handled automatically. These options consist of a single letter, from the list as follows:  

 \noindent
	_l_ - Force long form citation, regardless if the source has appeared before. \\
	_s_ - Force short form citation, even if this is the first cite to this source. \\
	_n_ - Force reporter and page number-only citation (for cases only). \\
	_I_ - If and only if ``\textit{id.}'' is used, force it to be capitalized. \\ 
	_i_ - If and only if ``\textit{Id.}'' is used, force it to be non-capitalized. \\
	_!_ - Record a cite at this location (and thus to the ToA / record Supra), but do not actually print anything. \\ 
	_*_ - Print the citation here, but do not record it to the table of authorities.  


\cmd{Id, \textbackslash{}id}[Pin Page][Cite the previous source]
 \begin{Example}
   \pincite{Marbury}{140}. \\
   \See \id[140]. \\
   \Id[141].
 \end{Example}

 \noindent The effect is to repeat the previous cite, including the previous pin page (unless the optional argument is used
 to cite a different page). _\Id_ should generally be followed by a period or other punctuation, as the trailing period 
 will not be added automatically.

\cmd{citetext}{Arbitrary text}[Automatically place law review citations in footnotes.]

 \noindent The purpose of this command is primarily to work with automatic footnotes in law review mode. 
 Whatever text is passed as its argument will automatically be put in a footnote, if it is not already in 
 a footnote (in which case _\citetext_ does nothing). If you are not using law review mode, this command 
 is not necessary. You would ordinarily put a space between the closing punctuation and the _\citetext_,
 the macro will automatically remove that space, if appropriate.

\cmd{citeclause}{Arbitrary Text}[Cite arbitrary text in an intra-sentence citation clause]

\noindent Example:

 \begin{quote}\texttt{It is the role of the judicial department to say what the law is \\ \textbackslash{}citeclause\{\textbackslash{}see \textbackslash{}cite\{Marbury\}\} and the present case is no exception.}\end{quote}

\noindent Result (Normal Mode):

 \begin{quote}It is the role of the judicial department to say what the law is, \see \cite[s]{Marbury}, and the present case is no exception.\end{quote}

\noindent Result (Law Review mode):

 \begin{quote}It is the role of the judicial department to say what the law is,\footnote{\See \cite[s]{Marbury}} and the present case is no exception.\end{quote}

 \noindent This function is probably not necessary for production use, but allows the samples to have one source code for both
 normal and lawreview mode. When using a citation clause in the middle of a sentence, use _\citeclause_ at the location,
 WITHOUT any surrounding punctutation. In standard mode, _\citeclause_ will add surrounding commas, unless the following
 character is a period, in which case it add a preceding comma and leaves the period to follow, as-is.

 In lawreview mode, _\citeclause_ will insert a preceding comma, and insert a footnote with the cited text immediately 
 after the comma, with no punctuation following -- UNLESS the following character is a period. In the latter case, 
 _\citeclause_ will add only a preceding period, with the footnote immediately following. Also, if the first token of the 
 citation clause is one of %% the pre-defined lowercase citation signals (_\see, \cf_ etc.), it will be automatically 
 converted to the uppercase equivalent.

 Note that in either case, _\citeclause_ is not able to properly handle a citation for a quote, in which case the punctuation
 should go inside the closing ''. In that case, you will either need to make the change manually when changing between
 normal and lawreview modes, or use the _\PeriodOrComma_ macro. The latter is a Period in lawreview, and a Comma normally. 

 If you're thinking this is more hassle than its worth, you're right. Feel free to just write out the correct form manually.

%\cmd{@autofootnote}{Put argument in a footnote, if in lawreview mode}

~\par

\noindent
\begingroup\raggedright\hyphenpenalty=10000\hangindent=1.5em
\texttt{\bf \textbackslash{}Reporter, \textbackslash{}ShortName, \textbackslash{}FullName, \textbackslash{}StartPage, \textbackslash{}Parenthetical, \textbackslash{}Authors, \textbackslash{}BookTitle, \textbackslash{}SrcType, \textbackslash{}SupraNote, \textbackslash{}LastNote} \par\noindent
\endgroup
 Usage: \\
	_\Reporter_\arg{Short Name}, _\StartPage_\arg{Short Name}, \emph{etc.}

\begin{Example}
   \FullName{Steel Seizure Case} \\
   \Reporter{Marbury} \\
   \StartPage{Marbury} 
\end{Example}

 \noindent These functions are predominantly helper functions used elsewhere in the code; they print components of the source. If the field
 that you have selected is inapplicable to the source, latex will give an error. Most should be clear by their name; _\Prefix_ is the 
 leading number of a statute source, if applicable (such as the ``42'' in ``42 U.S.C. \S 1983''). _\SrcType_ is one of ``Case,'' ``Book,'' 
 ``Statute,'' or ``Other.'' _\SupraNote_ is the first footnote in which a source appears. _\LastNote_ is the most recent note
 in which a source appears.


 \cmd{SetIndexType}{Short Name}{New Index Type}[Change a source's destination index / table of authorities] 

 \noindent This command set the destination index for the source provided as their argument. The default index files are ``Case''
 ``Statute'' and ``Other,'' for cases, statutes, and everything else, respectively. _\SetIndexType_ can be used to alter the index file. 
 Note in particular that if _\SetIndexType_ is set to an empty string, indexing will be disabled for the given source. 
 The current index for a source can be queried with _\IndexType_. 


 \cmd{SetIndexName}{Short Name}{Name To Appear In Index}[Change the appearance of a source in the index / table of authorities]

 \noindent This can be used to provide additional detail in the Table of Authorities that would not be appropriate in the flow of the text.
 For example, consider the Federal Rules of Civil Procedure:

 _\newstatute{Fed. R. Civ. P.}{}_ \\
 \hspace*{\parindent}_\SetIndexName{Fed. R. Civ. P.}{Federal Rules of Civil Procedure !Rule }_

 \noindent The second line ensures that ``Federal Rules of Civil Procedure'' will be written long form in the Table, and that each rule will
 be a subentry under this heading, to be preceded by ``Rule.'' Currently, this has effect only for statutes.


~\par

\noindent
\begingroup\raggedright\hyphenpenalty=10000\hangindent=1.5em
\texttt{\bf \textbackslash{}See, \textbackslash{}Seealso, \textbackslash{}Seeeg, \textbackslash{}Seegenerally, \textbackslash{}Cf, \textbackslash{}Butsee, \textbackslash{}Butseeeg, \textbackslash{}Butcf, \textbackslash{}Compare, \textbackslash{}Contra, \textbackslash{}Accord} --- Introductory Signals 
\endgroup

 \begin{Example}
   \See \pincite{Marbury}{117}. \\
   \Seealso \pincite{Marbury}{120}. \\
   \Cf \pincite{Lochner}{602}. 
 \end{Example}

  \noindent These commands insert the italicized signal word in front of the citation,
  and cause the cite command to automatically handle correct capitalization of an ``\textit{id.},'' should one be used.
  Each command has both a capitalized (_\See_, _\Cf_, etc.) and a non-capitalized (_\see_, _\cf_, etc.) version.
  Also, in law review mode, they will correctly appear in the footnote with the citation, obviating the need to wrap 
  every source in a _\footnote_ command. 

\section{Miscellaneous}


 \noindent
_\S, \P_ --- Section and Paragraph symbols ``\S\unskip, \P\unskip,'' preserving following space.


 \noindent
_\ldots, \ldotss_ --- Non-breaking 3-dot and 4-dot Ellipses ``\ldots , \ldotss'' 

 \noindent
_\ellipsedotspacing_ --- Gap between dots in _\ldots_ (default \ellipsedotspacing) 


 \noindent
_\Ordinal, \ordinal_ --- Given an integer, convert into a textual ordinal number (through Tenth) 


 \noindent
_\availableat{URL}_ --- Provides a web citation in the text, using the URL package 


 \noindent
 _\makeandletter_ --- Allows ampersands to appear in the text.

 \noindent
 _\makeandtab_ --- Returns ampersands to their default definition.


\section{Configuration Options}

 \noindent _\CF, \BTF, \ATF, \BAF_ --- Fonts to be used in citations

 These define the fonts used for Cases, Book Titles, Article Titles, and Book Authors, respectively.
 In normal mode, the latter three are defined to _\em_, and _\BAF_ is empty (roman).
 In lawreview mode, cases are in roman font, Article Titles _\em_,  and Book Titles and Authors small-capped.

 These can be redefined as needed. Note that, because _\em_ is used, italics can be converted to underlines by including
 _\usepackage{ulem}_ at the start of the file.

~\par

\noindent {\bf\large Package Option _lawreview_} --- Switch to Law Review-type formatting 

 By adding this option to the package declaration, citations will display case titles in roman font, 
 while author and title are in small-caps. Citations will automatically be made into footnotes, by default 10pt. 
 If the source is combined with a introductory signal (\emph{See, Cf., etc.} ), the signal will also be 
 placed in the footnote. However, if a citation has multiple source in a string cite or parentheticals, 
 the entire footnote should be put into a LaTeX _\footnote_ command, or _\citetext_.
%
%\noindent Package Option _casesupra_ --- Use \emph{supra} in case short-form citations. 
%
% The Bluebook says that subsequent citations to a case should not use the word \textit{supra}
% to indicate that the case has been fully cited previously. However, in practice, this is 
% often seen, especially in Supreme Court opinions with their three parallel citations. 

\noindent _\indexglue, \indexpenalty_ --- Parameters for line-breaking Table of Authorities entries 

 In classes that create a table of authorities (such as _lawbrief.cls_), we encourage LaTeX to break
 long lines immediately after the Long Name, such that the citation is put on the next line
 (i.e. rather than putting a break in the middle of the citation). These two parameters control that,
 and generally will not need to be changed. The _\indexglue_ defaults to _0in plus 1.fil_, and _\indexpenalty_
 defaults to _-999_. 

\noindent _\maxsequentialids_ --- Maximum number of \textit{Id.}'s in a row

 After this number of sequential id's, we force a short cite for clarity. 
 Set it very large to disable this functionality.
 

\noindent _\forcelongevery_ --- Force a long citation after a source has not appeared in this many footnotes 

 The Bluebook has a ``5 Footnote Rule,'' in which a source that has not appeared in the last five footnotes is required
 to be provided as a long citation. Therefore, this parameter is set to 5 by default. Disable it by setting it to a large 
 value. Note that it has no effect unless in lawreview mode.
 
