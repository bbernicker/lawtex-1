\providecommand{\documentclassflag}{}
\documentclass[12pt,\documentclassflag]{FRAP_Brief}


%We use \newcase because the \emph will throw off parsing in \citecase
\newcase{Hosanna-Tabor I}{E.E.O.C. v. Hosanna-Tabor Evangelical Lutheran Church and School \emph{(Hosanna-Tabor I)}}
	{582 F. Supp. 2d}{881}{(E.D. Mich. 2008)}
\newcase{Hosanna-Tabor II}{E.E.O.C. v. Hosanna-Tabor Evangelical Lutheran Church and School \emph{(Hosanna-Tabor II)}}
	{597 F.3d}{769}{(6th Cir. 2010)}

%Saves me from writing out the full cite command for the district and circuit court opinions
\def\HTDist#1{\pincite{Hosanna-Tabor I}{#1}}
\def\HTApp#1{\pincite{Hosanna-Tabor II}{#1}}

\citecase{Rweyemamu v. Cote, 520 F.3d 198 (2d Cir. 2008)}
\citecase{McClure v. Salvation Army, 460 F.2d 553 (5th Cir. 1972)}
\citecase{Petruska v. Gannon Univ., 462 F.3d 294 (3d Cir. 2006)}
\citecase{Elvig v. Calvin Presbyterian Church, 375 F. 3d 951 (9th Cir. 2004)}
\citecase{Natal v. Christian and Missionary Alliance, 878 F.2d 1575 (1st Cir. 1989)}
\citecase{Lewis v. Seventh Day Adventists Lake Region Conf., 978 F.2d 940 (6th Cir. 1992)}
\citecase[Catholic Univ.]{E.E.O.C. v. Catholic Univ. of America, 83 F.3d 455 (D.C. Cir. 1996)}
\citecase{Presbyterian Church v. Hull Church, 393 U.S. 440 (1969)}
\citecase[Milivojevich]{Serbian Eastern Orthodox Diocese for the USA and Canada v. Milivojevich, 426 U.S. 696 (1976)}
\citecase{Kedroff v. St. Nicholas Cathedral of the Russian Orthodox Church in North America, 344 U.S. 94 (1952)}
\citecase{Larson v. Valente, 456 U.S. 228 (1982)}
\citecase{Watson v. Jones, 80 U.S. 679 (1872)}
\citecase{Thomas v. Review Bd., 450 U.S. 707 (1981)}
\citecase{Walz v. Tax Comm'n of City of New York, 397 U.S. 664 (1970)}
\citecase{Everson v. Bd. of Ed. of Ewing, 330 U.S. 1 (1947)}
\citecase{Arbaugh v. Y&H Corp., 546 U.S. 500 (2006)}

\newbook{Prosser and Keaton}{William Lloyd Prosser & W. Page Keaton}{The Law of Torts}{(2nd ed. 1953)}


\newarticle{Note, \emph{The Ministerial Exception}}{Note}{The Ministerial Exception To Title VII: The Case for a Deferential Primary Duties Test}{121 Harv. L. Rev.}{1776}{(2009)}{}

\newstatute{Fed. R. Civ. P.}{}
\SetIndexName{Fed. R. Civ. P.}{Federal Rules of Civil Procedure !Rule }

\newmisc{J.A.}{J.A. \pin{}{}}
\SetIndexType{J.A.}{}

%The following line add ``U.S. Const. Amend. I'' to the Statutes index section,
% uses the aa@ prefix to order it first in the list, and marks it passim
% \index{Statute}{aa@\textsc{U.S. Const.} amend. I|idxpassim}
{\makeatletter % it's not necessary to make at a letter if this is already true. But calling \makeatletter in a brace-delimited block is good practice because it always works and does not affect the rest of the document.
  \newstatute[0@1stamend]{\textsc{U.S. Const.} amend. I}{} % put at place 0, call it 1stamend
  \newstatute[42@42 USC]{Title 42, United States Code}{}
  \newstatute[28@28 USC]{Title 28, United States Code}{}
}


%Set the information for the title page (later produced by \makefrontmatter)
\docket{No. 21-1234}
\circuit{First}
\petitioner{Hosanna-Tabor Evangelical Church and School}
\respondent{Equal Opportunity Employment Commission}
\briefposture{On Remand from the \\ Supreme Court of the United States}
\nameofbrief{Brief for Petitioner}
\author{Brendan Bernicker}
\authorrole{Counsel for Petitioner}
\authorfirm{Yale Law School}
\authoraddress{127 Wall Street \\ New Haven, CT 06510}
\authorphone{(610) 203-0293}
\authoremail{brendan.bernicker@yale.edu}
\filingdate{30th day of August, 2021} %format is important for certificate of service
\pythonwordcount{PYTHON WILL INPUT THE WORD COUNT HERE}

\disclosurestatementrequired{0} %Yes(1)/No(0), tells the computer whether to create page for it
\disclosurestatement{{If required, the disclosure statement goes here.}}


\begin{document}
%This commands creates the title page, table of contents, and table of authorities
\makefrontmatter

%Sets the formatting for the entire document after the front matter
\parindent=2em
\setlength{\parskip}{1.25ex plus 2ex minus .5ex}
\setstretch{2}  %this is double spacing

\section{Summary of Argument}

The First Amendment provides that "Congress shall make no law respecting an establishment of religion, or prohibiting the free exercise thereof." The Establishment and Free Exercise Clauses are rooted in the Founders' understanding that "the best interest of a society require[s] that the minds of men always be wholly free." Therefore, these Religion Clauses prohibit the "excessive entanglement" of church and state. In particular, "there is substantial danger that the State will become entangled in essentially religious controversies or intervene on behalf of groups espousing particular doctrinal beliefs" when hearing internal church disputes. Therefore, lower courts have developed the "ministerial exception," by which the courts will not examine a church's decision to hire or fire its ministers.  

\section{Argument}

\subsection{Standard of Review}

The parties contest the appropriate standard of review regarding the ministerial exception. Petitioner Hosanna-Tabor requests this Court to consider the claim as a challenge to subject-matter jurisdiction as did both the district court and the Court of Appeals. Thus the legal conclusions of the lower courts are reviewed de novo, and factual findings are affirmed unless clearly erroneous.\footnote{This is almost always true. It is stupid to write this in every brief and opinion.}

\textbf{The district court's judgment was not \textit{specifically} in response to a 12(b)(1) motion}. Instead, the court issued its judgment in response to a Rule 56 motion for summary judgment. Typically, an entry of summary judgment is appropriate when "there is no genuine dispute as to any material fact and the movant is entitled to judgment as a matter of law."

\subsection{The Ministerial Exception is a necessary consequence of the First Amendment's guarantee of Freedom of Religion}

This ministerial exception, however, is not merely a prudential doctrine of judicial restraint, but a necessary consequence of the Constitution's commitment to religious freedom.\footnote{This committment to religious freedom is reflected in the First Amendment's twin promises of free-exercise and non-establishment.} While this Court has not spoken directly to the ministerial exception, the doctrine follows directly from this Court's interpretation of the Establishment and Free Exercise Clauses. 

\makeendmatter 

\end{document}
