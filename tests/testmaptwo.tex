\providecommand{\documentclassflag}{}
\documentclass[12pt,\documentclassflag]{lawbrief} 

\usepackage[margin=1in]{geometry}
\usepackage{newcent,microtype}
\usepackage{setspace,xcolor}
\usepackage[hyperindex=false,linkbordercolor=white,pdfborder={0 0 0}]{hyperref}
\usepackage{test}
\usepackage{showexpl} % LTXexample blocks
\usepackage{xstring}
\usepackage{listings}
\usepackage{etoolbox}
\begin{document}

\section{Testing maptwo}

We test a looping macro that applies a two-argument command to each
pair of two lists. Macro \lstinline|\maptwo| is called with five
parameters: the input list separator, the output list separator, the
macro to apply to each pair, and the two lists.

\begin{LTXexample}
  \long\def\maptwo#1#2#3#4#5{{% create a block to restrict the scope
    \def\endlist{\endlist}% mark the end of the list
    \makeatletter
    % maptwo,;\mac{a,b,c}{1,3,3} ->
    % \@maptwo{a},{b,c},\endlist{1},{2,3},\endlist
    % note that args 1 and 3 are the first elements, and 2 and 4 are
    % the rest. Comma \endlist serves as an end-of-list marker
    \long\def\@maptwo##1#1##2\endlist##3#1##4\endlist{%
      \ifx\empty##1\empty\relax\else\ifx\empty##3\empty\relax\else{}#3{##1}{##3}\fi\fi%
      \ifx\empty##2\empty\relax\else\ifx\empty##4\empty\relax\else%
      #2\@maptwo##2\endlist{}##4\endlist\fi\fi}
    \@maptwo#4#1\endlist{}#5#1\endlist}}
  \def\hello#1#2{hello #1 #2}

  Testing \textbackslash{}maptwo:

  Call \lstinline|\maptwo| with input list separator semicolon,
  output list separator \lstinline|\par\vspace{5mm}|, macro \lstinline|\hello|, and lists
  \lstinline|one;two; three;four| and \lstinline|a;b;c|.
  
  \maptwo;{\par\vspace{5mm}}{\hello}{one;two; three;four}{a;b;c}
  
\end{LTXexample}

\section{Defining a macro in a macro}

Basic: 

\begin{LTXexample}
  \def\mklst#1{\def#1{one,two,three}}
  \mklst{\hello}
  hello is now: 
  \hello 
\end{LTXexample}

If we want to use a macro in its own definition, we need to edef it,  
otherwise \TeX\ gets confused and goes into an infinite loop.  

 \begin{LTXexample}
  \def\mklst#1{\edef#1{#1,one,two,three}}
  \def\hello{zero}
  \mklst{\hello}
  hello is now:   \hello 
\end{LTXexample}

\begin{LTXexample}
  \def\addtolst#1#2,{\ifx\relax#2\relax\def\next{\relax}%
    List is #1\par\else Defining #1\par\edef#1{#1;#2}%
  \def\next{\addtolst{#1}}\fi\next}

\def\mylst{zero}

\addtolst{\mylst}one,two,three,,
\end{LTXexample}

\section{Undoing maptwo}

We can undo \textbackslash{}maptwo by taking a list and decomposing it
into two lists.

\begin{LTXexample}
  \def\undoMapTwo#1#2#3 #4,{% Expects a comma separated list with
                            % ending flag being two commas separated
                            % by a space.
    \def\next{\undoMapTwo{#1}{#2}}%
    \ifx\relax#3\relax Lists are: (#1) and (#2)\par\def\next{\relax}% Stop
    \else\ifx\empty#1\empty Initializing the macros\par\edef#1{#3}\edef#2{#4}% first time
    \else Adding to the list #1 \edef#1{#1, #3} Adding to the list #2\par \edef#2{#2, #4}%
    \par\fi\fi\next}

  \def\one{}\def\two{} % TeX complains if these are not defined.
  \undoMapTwo\one\two 1 one,2 two,3 three, ,

  List one is \one.

  List two is \two.
  
\end{LTXexample}

We can apply a macro to every element in a list and gather the results 
in a list. We can roll our own, or we can use etoolbox's 
\textbackslash{}DeclareListParser. 

\section{Left Trim}

Here we define a left trim macro which removes leading spaces. It was
intended to be used to strip out spaces in case citations, but it 
turns out that this is not necessary.

\begin{LTXexample}
  \def\ltrimtrace#1{{Processing (#1)\par\IfEq{#1}{ }{\def\next{ltrim}}{\def\next{#1}}\next}}
  \def\ltrim#1{{\IfEq{#1}{ }{\def\next{ltrim}}{\def\next{\relax #1}}\next}}

  space test
  
  \def\hello{    hello}

  trimmed hello: (\ltrim    hello) 

  trimmed \textbackslash{}hello: (\expandafter\ltrim\hello)
  
  trimmed \textbackslash{}hello: (\ltrim   \hello) 

  no space test
  
  \def\hello{hello}

  trimmed hello: (\ltrim    hello)

  trimmed \textbackslash{}hello: (\expandafter\ltrim\hello) 
\end{LTXexample}

\section{Map}

We can apply a macro to every element in a list and gather the results 
in a list. We can roll our own, but prefer to use etoolbox's 
\textbackslash{}DeclareListParser. 


\begin{LTXexample}

Define books, pages, getbook, getpage, and processCites:
  
  \def\books{}
  \def\pages{}
  \def\getbook#1 #2 #3;{#1 #2}
  \def\getpage#1 #2 #3;{#3}
  \DeclareListParser{\processCites}{;}\par

  Redefine do:
  
  \renewcommand\do[1]{%
    \edef\book{\getbook#1;}%
    \edef\page{\getpage#1;}%
    \IfEq{\books}{}{%
    \edef\books{\book}\edef\pages{\page}%
    }{% else books is not empty
    \edef\books{\books;\book}%
    \edef\pages{\pages;\page}%
    }%
    % \par books is (\books)\par pages is (\pages)\par \vspace{5mm}% debug
    }
    
    processCites:
    
    \processCites{123 Mich 456; 234 NW2d 567}
    
    books: (\books); pages: (\pages)\par
  \end{LTXexample}

  \section{etoolbox list tools}
  
  \begin{LTXexample}
    \renewcommand\do[1]{hello (#1)\par}
    \docsvlist{one, two, three, four,five}
    \renewcommand\do[1]{\listadd{\hellolist}{hello (#1)}}
    \docsvlist{one, two, three, four,five}
    \providecommand\listjoin[2][,]{{%
        \makeatletter\def\separator@{\empty}%
        \renewcommand\do[1]{\separator@##1\def\separator@{#1}}%
        \dolistloop{#2}}}

    Test \textbackslash listjoin:
    \listjoin{\hellolist}

    \listadd{\newlist}{a}
    \listadd{\newlist}{a}
    \listadd{\newlist}{a}
    \listadd{\newlist}{a}
    newlist = (\listjoin{\newlist}) 
  \end{LTXexample}

  % following is an unsuccessful attempt to write a listpop macro.
  
      % \edef\listseparator{{\catcode`\|=3|}}
    % \newcommand\endlist{\endlist}
    \newcommand\poplist[1]{{%
    % \def\poplist#1{{%
        \def\poplisti##1\listseparator##2\endlist{##1}%
        \expandafter\poplisti#1\listseparator\endlist}}
    
    % \poplist{\hellolist} 

    

    % \newcommand\poplist[1]{{%
    %     \def\first{\relax}%
    %     \def\r{\relax}
    %     \renewcommand\do[1]{%
    %       proccessing ##1\par%
    %       \ifx\first\r%
    %       Setting first%
    %       \edef\first{##1} (\first)\par\fi}%
    %     Doing list\par%
    %     \dolistloop{#1}%
    %     first is (\first)\par%
    %     % \listremove{#1}{\first}
    %     \first%
    %   }}
    % \textbackslash poplist \textbackslash hellolist: 
    % (\poplist\hellolist) 
    % \listremove\hellolist{hello (one)}

    % \textbackslash hellolist (\listjoin\hellolist) 

    

\end{document}

