\providecommand{\documentclassflag}{}
\documentclass[12pt,\documentclassflag]{lawbrief} 

\usepackage[margin=1in]{geometry}
\usepackage{newcent,microtype}
\usepackage{setspace,xcolor}
\usepackage[hyperindex=false,linkbordercolor=white,pdfborder={0 0 0}]{hyperref}
\usepackage{test}
\usepackage{showexpl}
\usepackage{xstring}

\begin{document}

\section{Testing Test Framework}

We test the testing framework in this document. First we test
`\textbackslash{}StrCompare` to understand its output, then we go ahead and test the
`\textbackslash{}assert` macro.

\subsection{StrCompare}

The `StrCompare` macro takes two string arguments and prints the position of first character that
is different. If the two strings are the same, it returns 0.  It has two modes, `\comparestrict` and
`\comparenormal`. The normal mode will only match to the shorter
string, returning 0 if the shorter string is a prefix.
  
\begin{LTXexample}
  \StrCompare{1234}{01234} = 1, $1\ne0$\par
  {\comparenormal
    \StrCompare{1234}{12345}}=0, normal mode accepts prefixes\par
  {\comparestrict
    \StrCompare{1234}{12345}}=5, strict mode flags prefixes\par 
  \StrCompare{1234x}{12345}=5, $x\ne5$\par
\end{LTXexample}

We can also do simple macro substitutions in `StrCompare`.

\begin{LTXexample}
  \def\a{1234}
  \StrCompare{\a}{1234}=0, $1=1$\par
  \StrCompare{\a}{01234}=1, $1\ne0$\par
  \StrCompare{right}{wrong}=1, $r\ne{}w$\par
  \def\a{right}
  \StrCompare{\a}{wrong}=1, $r\ne\ w$\par
\end{LTXexample}

We can use macros that take arguments inside `StrCompare`.

\begin{LTXexample}
  \def\a#1{#1 world}
  \StrCompare{\a{hello}}{hello world}
\end{LTXexample}

We can also use \LaTeX\ `newcommand` in `StrCompare`.
\begin{LTXexample}
  \newcommand{\aaa}[1]{#1 world}
  \StrCompare{\aaa{hello}}{hello world}
\end{LTXexample}

But macros with optional arguements won't work
in `StrCompare`.

\begin{LTXexample}
  \newcommand{\aaaa}[1][hello]{#1 world}
  \aaaa\par\aaaa[hi]\par
  % \StrCompare{\aaaa[hello]}{hello world} % this raises an error, 
  % undefined control sequence
\end{LTXexample}

Macros with multiple arguments work in `StrCompare`.
\begin{LTXexample}
  \newcommand{\aaab}[2]{#1 #2}
  \aaab{hello}{world}\par 
  \StrCompare{\aaab{hello}{world}}{hello world} $=0$
\end{LTXexample}

\subsection{assert}

We can assert equality between simple strings. 

\begin{LTXexample}
  \assert{right}{wrong}
  \assert{right}{right}
  \assert[right and wrong]{right}{wrong}
  \assert[just right]{right}{right}
\end{LTXexample}

But \TeX complains if we try to use formatting in an assertion. 
  
\begin{LTXexample}
  \def\a{\emph{hello}}
  \def\b{\emph{hello }}
  % \assert{\a}{\b} % TeX complains:
%   I've run across a `}' that doesn't seem to match anything.
% For example, `\def\a#1{...}' and `\a}' would produce
% this error. If you simply proceed now, the `\par' that
% I've just inserted will cause me to report a runaway
% argument that might be the root of the problem. But if
% your `}' was spurious, just type `2' and it will go away.
\end{LTXexample}  

Simple macros with or without arguments are ok.

\begin{LTXexample}
  \def\a{hello }
  \assert{hello world}{\a world}

  \def\a#1{hello #1}
  \assert{\a{world}}{hello world}
  
  \def\a#1#2{#1 #2}
  \assert{\a{hello}{world}}{hello world}

  \newcommand\testAssertA{hello world}
  \assert{\testAssertA}{hello world}

  \newcommand\testAssertB[1]{#1 world}
  \assert{\testAssertB{hello}}{hello world}

  \newcommand\testAssertC[2]{#1 #2}
  \assert{\testAssertC{hello}{world}}{hello world}

\end{LTXexample}

But using `\textbackslash{}assert` with a `\textbackslash{}newcommand` with an optional argument
causes \TeX\ to throw an error.

\begin{LTXexample}
  \newcommand\testAssertD[1][hello]{#1 world}
  % \assert{\testAssertD}{hello world} % error 
  % TeX encountered an unknown command name. 
\end{LTXexample}

We can't use a macro defined with a `\textbackslash{}newcommand` in `\textbackslash{}assert`.

\begin{LTXexample}
  % \DeclareRobustCommand\testAssertE[1][hello]{#1 world}% doesn't work 
  \newcommand\testAssertE[1][hello]{#1 world}% doesn't work 
  \testAssertE 
  
  \testAssertE{}
  % \def\testResultE{\testAssertE\relax}% doesn't work 
  % \edef\testResultE{\protect\testAssertE[hello]}% doesn't work:
  % testAssertE doesn't match its definition.
  
  % \assert{\testResultE{}}{hello world} % error
\end{LTXexample}

\end{document}


