\providecommand{\documentclassflag}{}
\documentclass[12pt,\documentclassflag]{lawbrief} 

\usepackage[margin=1in]{geometry}
\usepackage{newcent,microtype}
\usepackage{setspace,xcolor}
\usepackage[hyperindex=false,linkbordercolor=white,pdfborder={0 0 0}]{hyperref}
\usepackage{test}
\usepackage{showexpl} % LTXexample blocks 
\usepackage{trace}

%%Cases
% \makeandletter

\begin{document}

\section{Testing the bluebook macros}

% \subsection{Testing countListSeparators}
% \begin{LTXexample}
%   \def\mymac#1#2{(#1)(#2)}
%   \assert{mymac}{\mymac{hello there}{danny}}{(hello there)(danny)}
%   \assert{mymac}{\mymac{\mymac{hello}{there}}{danny}}{((hello)(there))(danny)}
% \end{LTXexample}


% \subsection{Exploring ifcase}
% \begin{LTXexample}
%   \def\testcase#1{\ifcase#1 0: (#1)\or 1: (#1)\or 2: (#1)\or 3: (#1)\or 4: 
%     (#1)\or 5: (#1)\else else (#1)\fi\par}

%   \testcase{0}
%   \testcase{1}
%   \testcase{2}
%   \testcase{3}
%   \testcase{10}

%   \edef\out{\testcase{0}}\out 
%   \edef\out{\testcase{1}}\out 
%   \edef\out{\testcase{2}}\out 
%   \edef\out{\testcase{3}}\out 
%   \edef\out{\testcase{10}}\out 

%   \StrCount{}{1}[\c]\testcase\c 
%   \StrCount{1}{1}[\c]\testcase\c 
%   \StrCount{11}{1}[\c]\testcase\c 
%   \StrCount{111}{1}[\c]\testcase\c 
%   \StrCount{1111}{2}[\c]\testcase\c 
%   \StrCount{11111}{2}[\c]\testcase\c 

% \end{LTXexample}

% \subsection{Test shortname}
% \begin{LTXexample}
%   shortcasename = (\deriveShortCaseName{Ashcroft v. American Civil
%     Union})\par
  
%   \deriveShortCaseName[\name]{Ashcroft v. American Civil Union}
%   \assert{short name v.}{\name}{Ashcroft}\par

%   \deriveShortCaseName[\name]{Ashcroft v American Civil Union}
%   \assert{short name v}{\name}{Ashcroft}\par 
  
%   \deriveShortCaseName[\name]{Ashcroft vs. American Civil Union}
%   \assert{short name vs.}{\name}{Ashcroft vs. American Civil Union}\par
% \end{LTXexample}

\subsection{Test long citations}
Test new builtin citations. 
\begin{LTXexample}
  \citecase{Ashcroft v. American Civil Liberties Union, 535 U.S. 564 (2002)}\par
  \assert{Checking first reporter}{\Call{Ashcroft@ReporterA}}{535 U.S.}\par
  \assert{Checking second reporter}{\Call{Ashcroft@ReporterB}}{} \par
  \assert{Checking third reporter}{\Call{Ashcroft@ReporterB}}{} \par
  \assert{1 citation}{\bbCaseLongCitations{Ashcroft}{}}{535 U.S. 564 (2002)}\par
  \assert{1 citation with page}{\bbCaseLongCitations{Ashcroft}{12}}{535 U.S. 564, 12 (2002)}\par
\end{LTXexample}

Two reporters. 
\begin{LTXexample}
  \citecase{Two v. X, 123 Mich 234; 345 NW2d 456 (2010)}\par 
  \assert{2 citation}{\bbCaseLongCitations{Two}{}}%
  {123 Mich 234; 345 NW2d 456 (2010)}\par 

  \assert{2 citation with 1 page}{\bbCaseLongCitations{Two}{89}}%
  {123 Mich 234, 89; 345 NW2d 456 (2010)}\par 

  \assert{2 citation with 2 pages}{\bbCaseLongCitations{Two}{89; 356}}%\par 
  {123 Mich 234, 89; 345 NW2d 456, 356 (2010)}\par 

\end{LTXexample}

Three reporters. 
\begin{LTXexample}
  \citecase{Three v. X, 123 Mich 234; 345 NW2d 456; 789 MyRep 456 (2010)}\par 
  \assert{3 citation}{\bbCaseLongCitations{Three}{}}%
  {123 Mich 234; 345 NW2d 456; 789 MyRep 456 (2010)}\par 

  \assert{3 citation with 1 page}{\bbCaseLongCitations{Three}{1}}%\par 
  {123 Mich 234, 1; 345 NW2d 456; 789 MyRep 456 (2010)}\par 

\assert{3 citation with 2 pages}{\bbCaseLongCitations{Three}{1; 2}}%
{123 Mich 234, 1; 345 NW2d 456, 2; 789 MyRep 456 (2010)}\par 

\assert{3 citation with 3 pages}{\bbCaseLongCitations{Three}{1; 2; 3}}%
{123 Mich 234, 1; 345 NW2d 456, 2; 789 MyRep 456, 3 (2010)}\par 
\end{LTXexample}

Check Slip Opinions. 
\begin{LTXexample}
  \let\reporters\empty\let\startpages\empty 
  \parseCitations \_\_ Mich \_\_; X X X; \par 
  \assert{slip reporter}{\reporters}{\_\_ Mich}\par 
  \assert{slip page}{\startpages}{\_\_}\par 
  \citecase{Ashcroft v. American Civil Liberties Union, \_\_\_  U.S. \_\_\_ (2002)}\par 
  \assert{slip op}{\bbCaseLongCitations{Ashcroft}{}}%
  {\_\_\_ U.S. \_\_\_ (2002)}

  \assert{slip op with page}{\bbCaseLongCitations{Ashcroft}{slip op at 3}}%
  {\_\_\_ U.S. \_\_\_, \_\_\_ (2002); slip op at 3}\par 
\end{LTXexample}

\subsection{Test short citations}
Test new builtin citations. 
\begin{LTXexample}
  \citecase{Ashcroft v. American Civil Liberties Union, 535 U.S. 564 (2002)}\par
  \assert{1 citation}{\bbCaseShortCitations{Ashcroft}{}}{}\par
  \assert{1 citation with page}%
  {\bbCaseShortCitations{Ashcroft}{12}}%
  {535 U.S.\penalty100\ at~12}\par
\end{LTXexample}

Two reporters. 
\begin{LTXexample}
  \citecase{Two v. X, 123 Mich 234; 345 NW2d 456 (2010)}\par
  \assert{2 citation}{\bbCaseShortCitations{Two}{}}{}\par
  \assert{2 citation with 1 page}{\bbCaseShortCitations{Two}{89}}%
  {123 Mich\penalty100\ at~89}\par

  \assert{2 citation with 2 pages}{\bbCaseShortCitations{Two}{89; 356}}%\par
  {123 Mich\penalty100\ at~89; 345 NW2d\penalty100\ at~356}\par

\end{LTXexample}

Three reporters. 
\begin{LTXexample}
  \citecase{Three v. X, 123 Mich 234; 345 NW2d 456; 789 MyRep 456 (2010)}\par
  \assert{3 citation}{\bbCaseShortCitations{Three}{}}{}\par%

  \assert{3 citation with 1 page}{\bbCaseShortCitations{Three}{1}}%\par
  {123 Mich\penalty100\ at~1}\par

\assert{3 citation with 2 pages}{\bbCaseShortCitations{Three}{1; 2}}%
{123 Mich\penalty100\ at~1; 345 NW2d\penalty100\ at~2}\par

\assert{3 citation with 3 pages}{\bbCaseShortCitations{Three}{1; 2; 3}}%
{123 Mich\penalty100\ at~1; 345 NW2d\penalty100\ at~2; 789 MyRep\penalty100\ at~3}\par
\end{LTXexample}

Check Slip Opinions. 
\begin{LTXexample}
  \citecase{Ashcroft v. American Civil Liberties Union, \_\_\_  U.S. \_\_\_ (2002)}\par
  \assert{slip op}{\bbCaseShortCitations{Ashcroft}{}}{}\par%
  \assert{slip op with page}{\bbCaseShortCitations{Ashcroft}{slip op at 3}}%
  {\_\_\_ U.S.\penalty 100\ at~\_\_\_; slip op at 3}\par

\end{LTXexample}

\subsection{Test citecase}
\begin{LTXexample}
  \citecase{x v y, 12 ny2d 34 (2015)}
  \citecase[X]{X v Y, 12 ny2d 34 (2015)}
  % \citecase[X]{X v Y, 12 ny2d 34 (2015)} % throws error, X is
  % already defined.
\end{LTXexample}  

\subsection{Test FullName}

\begin{LTXexample}
\citecase{Ashcroft v. American Civil Liberties Union, 
    535 U.S. 564 (2002)}
\assert{FullName}{(\FullName{Ashcroft})}{(Ashcroft v. American Civil Liberties Union)}

\edef\x{(\FullName{Ashcroft})}
\assert{edef full name}{\x}{(Ashcroft v. American Civil Liberties Union)}

Check full name w/o parens 

\assert{FullName}{\FullName{Ashcroft}}{Ashcroft v. American Civil Liberties Union}

\end{LTXexample}

\subsection{Test long cite}
\begin{LTXexample}
  \citecase{Ashcroft v. American Civil Liberties Union, 
    535 U.S. 564 (2002)}
  \citecase{Bashcroft v. American Civil Liberties Union, 
    535 U.S. 564; 123 Mich 234 (2002)} % semicolon ; separate parallel citations 
  {
    \let\CF\empty% disable emphasis of case name, we want all text 
    \assert{long cite}{\LongCite{Ashcroft}{}}{\CF{Ashcroft v. American Civil Liberties Union}, 535 U.S. 564 (2002)}
    \assert{long cite}{\LongCite{Ashcroft}{2}}%
    {\CF{Ashcroft v. American Civil Liberties Union}, 535 U.S. 564, 2 (2002)}
  } 
\end{LTXexample}

Test cites with parallel citations
\begin{LTXexample}
  \citecase{Two v. X, 123 Mich 234; 345 NW2d 456 (2010)}\par 
  {
    \let\CF\empty% disable emphasis of case name, we want all text 
    \assert{2 citation}{\LongCite{Two}{}}%
    {\CF{Two v. X}, 123 Mich 234; 345 NW2d 456 (2010)}\par 
    \assert{2 citation with 1 page}{\LongCite{Two}{89}}%
    {\CF{Two v. X}, 123 Mich 234, 89; 345 NW2d 456 (2010)}\par 
    \assert{2 citation with 2 pages}{\LongCite{Two}{89; 356}}%\par 
    {\CF{Two v. X}, 123 Mich 234, 89; 345 NW2d 456, 356 (2010)}\par 
  }
\end{LTXexample}

Three reporters. 
\begin{LTXexample}
  \citecase{Three v. X, 123 Mich 234; 345 NW2d 456; 789 MyRep 456 (2010)}\par 
  {
    \let\CF\empty% disable emphasis of case name, we want all text 
  \assert{3 citation}{\LongCite{Three}{}}%
  {\CF{Three v. X}, 123 Mich 234; 345 NW2d 456; 789 MyRep 456 (2010)}\par 
  \assert{3 citation with 1 page}{\LongCite{Three}{1}}%\par 
  {\CF{Three v. X}, 123 Mich 234, 1; 345 NW2d 456; 789 MyRep 456 (2010)}\par 
  \assert{3 citation with 2 pages}{\LongCite{Three}{1; 2}}%
  {\CF{Three v. X}, 123 Mich 234, 1; 345 NW2d 456, 2; 789 MyRep 456 (2010)}\par 
  \assert{3 citation with 3 pages}{\LongCite{Three}{1; 2; 3}}%
  {\CF{Three v. X}, 123 Mich 234, 1; 345 NW2d 456, 2; 789 MyRep 456, 3
    (2010)}\par
  }
\end{LTXexample}

Check Slip Opinions. 
\begin{LTXexample}
  \citecase{Ashcroft v. American Civil Liberties Union, \_\_\_ U.S. \_\_\_ (2002)}\par
  {
    \let\CF\empty% disable emphasis of case name, we want all text 
    \assert{slip op}{\LongCite{Ashcroft}{}}%
    {\CF{Ashcroft v. American Civil Liberties Union}, \_\_\_ U.S. \_\_\_ (2002)}
    \assert{slip op with page}{\LongCite{Ashcroft}{slip op at 3}}%
    {\CF{Ashcroft v. American Civil Liberties Union}, \_\_\_
      U.S. \_\_\_, \_\_\_ (2002); slip op at 3}\par
  }
\end{LTXexample}

\subsection{Test short cite}
\begin{LTXexample}
  \citecase{Ashcroft v. American Civil Liberties Union, 
    535 U.S. 564 (2002)}
  \citecase{Bashcroft v. American Civil Liberties Union, 
    535 U.S. 564; 123 Mich 234 (2002)} % semicolon ; separate parallel citations 
  {
    \let\CF\empty% disable emphasis of case name, we want all text 
    \assert{short cite}{\ShortCite{Ashcroft}{}}{\CF{Ashcroft}}
    \assert{short cite}{\ShortCite{Ashcroft}{2}}%
    {\CF{Ashcroft}, 535 U.S.\penalty100\ at~2}
  } 
\end{LTXexample}

Test cites with parallel citations
\begin{LTXexample}
  \citecase{Two v. X, 123 Mich 234; 345 NW2d 456 (2010)}\par 
  {
    \let\CF\empty% disable emphasis of case name, we want all text 
    \assert{2 short citation}{\ShortCite{Two}{}}%
    {\CF{Two}}\par 
    \assert{2 short citation with 1 page}{\ShortCite{Two}{89}}%
    {\CF{Two}, 123 Mich\penalty100\ at~89}\par 
    \assert{2 short citation with 2 pages}{\ShortCite{Two}{89; 356}}%\par 
    {\CF{Two}, 123 Mich\penalty100\ at~89; 345 NW2d\penalty100\ at~356}\par 
  }
\end{LTXexample}

Three reporters. 
\begin{LTXexample}
  \citecase{Three v. X, 123 Mich 234; 345 NW2d 456; 789 MyRep 456 (2010)}\par 
  {
    \let\CF\empty% disable emphasis of case name, we want all text 
  \assert{3 short citation}{\ShortCite{Three}{}}%
  {\CF{Three}}\par 
  \assert{3 short citation with 1 page}{\ShortCite{Three}{1}}%\par 
  {\CF{Three}, 123 Mich\penalty100\ at~1}\par 
  \assert{3 short citation with 2 pages}{\ShortCite{Three}{1; 2}}%
  {\CF{Three}, 123 Mich\penalty100\ at~1; 345 NW2d\penalty100\ at~2}\par 
  \assert{3 short citation with 3 pages}{\ShortCite{Three}{1; 2; 3}}%
  {\CF{Three}, 123 Mich\penalty100\ at~1; 345 NW2d\penalty100\ at~2; 789 MyRep\penalty100\ at~3}\par
  }
\end{LTXexample}

Check Slip Opinions. 
\begin{LTXexample}
  \citecase{Ashcroft v. American Civil Liberties Union, \_\_\_ U.S. \_\_\_ (2002)}\par
  {
    \let\CF\empty% disable emphasis of case name, we want all text 
    \assert{slip op}{\ShortCite{Ashcroft}{}}%
    {\CF{Ashcroft}}
    \traceon
    \assert{slip op with page}{\ShortCite{Ashcroft}{slip op at 3}}%
    {\CF{Ashcroft}, \_\_\_ U.S.\penalty100\ at~\_\_\_; slip op at 3}\par
    \traceoff
  }
\end{LTXexample}

\subsection{Testing cite and pincite}
Pincites and cites use assignments (let, def, edef) and index and as
far as I know, cannot be used in assert.

Testing cites and pincites cannot take place in LTXexample blocks.

  \citecase{A v. B, 1 Mich App 2 (2002)}
  \citecase{B v. B, 1 U.S. 2; 3 Mich\,App 4 (2002)}
  \citecase{C v. B, 1 U.S. 2; 3 Mich\,App 4; 5 Mich App 6 (2002)}
  Here are some pincites.
  \pincite{A}{2} and \par
  \pincite{B}{3} and \par
  \pincite{A}{4} and \par 
  \pincite{B}{2; 3} and \par
  \pincite{A}{2} and \par
  \pincite{C}{1} and 
  \pincite{C}{2} and 
  \pincite{C}{1; 3} and  
  \pincite{C}{1; 2; 3} and  
  And some cites  and \par
  \cite{A} and \par
  \cite{A} and \par

  \cite{A} and \par
  \cite{A} and \par
  \cite{A} and \par
  \cite{A} and \par
  \cite{A} and \par

\end{document}