\providecommand{\documentclassflag}{}
\documentclass[12pt,\documentclassflag]{lawbrief} 

\usepackage[margin=1in]{geometry}
\usepackage{newcent,microtype}
\usepackage{setspace,xcolor}
\usepackage[hyperindex=false,linkbordercolor=white,pdfborder={0 0 0}]{hyperref}
\usepackage{test}
\usepackage{showexpl} % LTXexample blocks 
\usepackage{trace}

%%Cases
% \makeandletter

\begin{document}

\section{Testing the bluebook macros}

% \subsection{Testing countListSeparators}
% \begin{LTXexample}
%   \def\mymac#1#2{(#1)(#2)}
%   \assert{mymac}{\mymac{hello there}{danny}}{(hello there)(danny)}
%   \assert{mymac}{\mymac{\mymac{hello}{there}}{danny}}{((hello)(there))(danny)}
% \end{LTXexample}


% \subsection{Exploring ifcase}
% \begin{LTXexample}
%   \def\testcase#1{\ifcase#1 0: (#1)\or 1: (#1)\or 2: (#1)\or 3: (#1)\or 4: 
%     (#1)\or 5: (#1)\else else (#1)\fi\par}

%   \testcase{0}
%   \testcase{1}
%   \testcase{2}
%   \testcase{3}
%   \testcase{10}

%   \edef\out{\testcase{0}}\out 
%   \edef\out{\testcase{1}}\out 
%   \edef\out{\testcase{2}}\out 
%   \edef\out{\testcase{3}}\out 
%   \edef\out{\testcase{10}}\out 

%   \StrCount{}{1}[\c]\testcase\c 
%   \StrCount{1}{1}[\c]\testcase\c 
%   \StrCount{11}{1}[\c]\testcase\c 
%   \StrCount{111}{1}[\c]\testcase\c 
%   \StrCount{1111}{2}[\c]\testcase\c 
%   \StrCount{11111}{2}[\c]\testcase\c 

% \end{LTXexample}

% \subsection{Test shortname}
% \begin{LTXexample}
%   shortcasename = (\deriveShortCaseName{Ashcroft v. American Civil
%     Union})\par
  
%   \deriveShortCaseName[\name]{Ashcroft v. American Civil Union}
%   \assert{short name v.}{\name}{Ashcroft}\par

%   \deriveShortCaseName[\name]{Ashcroft v American Civil Union}
%   \assert{short name v}{\name}{Ashcroft}\par 
  
%   \deriveShortCaseName[\name]{Ashcroft vs. American Civil Union}
%   \assert{short name vs.}{\name}{Ashcroft vs. American Civil Union}\par
% \end{LTXexample}

\subsection{Test parseCitations}
Test new builtin citations.

\begin{LTXexample}
  \citecase{Ashcroft v. American Civil Liberties Union, 535 U.S. 564 (2002)}

\assert{Checking first reporter}{\Call{Ashcroft@ReporterA}}{535 U.S.} 

\assert{Checking second reporter}{\Call{Ashcroft@ReporterB}}{} 

\assert{Checking third reporter}{\Call{Ashcroft@ReporterB}}{} 

% \traceon 
% \bbCaseLongCitations{Ashcroft}{}
% \traceoff 

\assert{1 citation}{\bbCaseLongCitations{Ashcroft}{}}%
{535 U.S. 564 (2002)}

\assert{1 citation with page}{\bbCaseLongCitations{Ashcroft}{12}}%
{535 U.S. 564, 12 (2002)}

\citecase{Two v. X, 123 Mich 234; 345 NW2d 456 (2010)}
\assert{2 citation}{\bbCaseLongCitations{Two}{}}%
{123 Mich 234; 345 NW2d 456 (2010)}

\assert{2 citation with 1 page}{\bbCaseLongCitations{Two}{89}}%
{123 Mich 234, 89; 345 NW2d 456 (2010)}

\assert{2 citation with 2 pages}{\bbCaseLongCitations{Two}{89; 356}}%
{123 Mich 234, 89; 345 NW2d 456, 356 (2010)}

\citecase{Three v. X, 123 Mich 234; 345 NW2d 456; 789 MyRep 456 (2010)}
\assert{3 citation}{\bbCaseLongCitations{Three}{}}%
{123 Mich 234; 345 NW2d 456; 789 MyRep 456 (2010)}

\assert{3 citation with 1 page}{\bbCaseLongCitations{Three}{1}}%
{123 Mich 234, 1; 345 NW2d 456; 789 MyRep 456 (2010)}

\assert{3 citation with 2 pages}{\bbCaseLongCitations{Three}{1; 2}}%
{123 Mich 234, 1; 345 NW2d 456, 2; 789 MyRep 456 (2010)}

\assert{3 citation with 3 pages}{\bbCaseLongCitations{Three}{1; 2; 3}}%
{123 Mich 234, 1; 345 NW2d 456, 2; 789 MyRep 456, 3 (2010)}

\end{LTXexample}

Check Slip Opinions.

\begin{LTXexample}
  \let\reporters\empty\let\startpages\empty
  \parseCitations \_\_ Mich \_\_; X X X; 
  \assert{slip reporter}{\reporters}{\_\_ Mich}
  \assert{slip page}{\startpages}{\_\_}
  \citecase{Ashcroft v. American Civil Liberties Union, \_\_\_  U.S. \_\_\_ (2002)}
  \assert{slip op}{\bbCaseLongCitations{Ashcroft}{}}%
  {\_\_\_ U.S. \_\_\_ (2002)}

  \assert{slip op with page}{\bbCaseLongCitations{Ashcroft}{slip op at 3}}%
  {\_\_\_ U.S. \_\_\_, \_\_\_ (2002); slip op at 3}

\end{LTXexample}

%   \subsection{Test Citations}
%   % \begin{LTXexample}
%   %   \def\testCitations#1#2#3#4#5#6#7{%
%   %     % 1: shortname 2: reporters, 3: start pages, 
%   %     % 4: pages, 5: slip suffix, 6: number of reporters, 7: number of 
%   %     % pages.
%   %     \edef\name{#1}%
%   %     \edef\reporters{#2}%
%   %     \edef\startpages{#3}%
%   %     \edef\pages{#4}%
%   %     \edef\slipsuffix{#5}%
%   %     \edef\numberOfReporters{#6}%
%   %     \edef\numberOfPages{#7}%
%   %     \long\def\check##1##2##3##4##5##6##7{%
%   %       % \par check 1(##1) 2(##2) 3(##3) 4(##4) 5(##5) 6(##6) 7(##7)\par%
%   %       % \par testing\par%
%   %       \expandafter\assert\expandafter{\name ~name}{\name}{##1}%
%   %       \expandafter\assert\expandafter{\name ~reporters}{\reporters}{##2}%
%   %       \expandafter\assert\expandafter{\name ~startpages}{\startpages}{##3}%
%   %       \expandafter\assert\expandafter{\name ~pages}{\pages}{##4}%
%   %       \expandafter\assert\expandafter{\name ~slipsuffix}{\slipsuffix}{##5}%
%   %       \expandafter\assert\expandafter{\name ~numberOfReporters}{\numberOfReporters}{##6}%
%   %       \expandafter\assert\expandafter{\name ~numberOfPages}{\numberOfPages}{##7}%
%   %     }%
%   %     % testCitations 1(#1) 2(#2) 3(#3) 4(#4) 5(#5) 6(#6) 7(#7)\par%
%   %     \bbCaseCitationEval{\name}{\pages}{\check}%
%   %   }%

%   %   \citecase{A1 v. B, 1 U.S. 2 (2002)}\par
%   %   \citecase{B1 v. B, 1 U.S. 2; 3 Mich App 4 (2002)}
%   %   \citecase{C1 v. B, 1 U.S. 2; 3 Mich App 4; 5 Mich App 6 (2002)}
%   %   \assert{A1-long cite}{bb
%   %   \testCitations{A1}{1 U.S.}{2}{3}{\empty}{1}{1}\par
%   %   \testCitations{B1}{1 U.S.; 3 Mich App}{2; 4}{3}{}{2}{1}
%   %   \testCitations{C1}{1 U.S.; 3 Mich App; 5 Mich App}{2; 4; 6}{3}{}{3}{1}
 
%   % \end{LTXexample}
%   \subsection{Test case parsing}
% \begin{LTXexample}
%   \def\cont#1#2#3#4#5{%
%     \par%
%     short name: (#1)\par%
%     case name: (#2)\par%
%     reporters: (#3)\par%
%     start pages: (#4)\par%
%     parenthetical: (#5)\par%
%     \vspace{5mm}
%   }
%   \long\def\checkparse#1#2#3#4#5#6#7{{%1,2: args to parseCase, 3-7:
%       % expected
%       % in extra blocks to avoid contamination
%       \long\def\check##1##2##3##4##5{%
%         \assert{short name}{##1}{#3}\par%
%         \assert{case name}{##2}{#4}\par%
%         \assert{reporters}{##3}{#5}\par%
%         \assert{start pages}{##4}{#6}\par%
%         \assert{parenthetical}{##5}{#7}\par%
%       }% end check
%       \long\def\checkShortname##1##2##3##4##5{%
%         \assert{derived short name}{##1}{#3}\par%
%         \vspace{5mm}%
%       }% end checkShortname
%       parsing with arguments (#1) and (#2)\par
%       \parseCase[#1]{#2}{\check}%
%       \parseCase{#2}{\checkShortname}%
%     }}%end checkparse
 

%   \checkparse{Ashcroft}{Ashcroft v. American Civil Liberties Union, 
%     535 U.S. 564 (2002)}{Ashcroft}{Ashcroft v. American Civil 
%     Liberties Union}{535 U.S.}{564}{(2002)}\par 
%   \checkparse{Ashcroft}{Ashcroft v. American Civil Liberties Union, 535 U.S. 564; 
%     123 nw2d 456 (2002)}{Ashcroft}{Ashcroft v. American Civil 
%     Liberties Union}{535 U.S.; 123 nw2d}{564; 456}{(2002)}

% \end{LTXexample}

% \subsection{Test citecase}

% \begin{LTXexample}
%   \citecase{x v y, 12 ny2d 34 (2015)}
%   \citecase[X]{X v Y, 12 ny2d 34 (2015)}
%   % \citecase[X]{X v Y, 12 ny2d 34 (2015)} % throws error, X is
%   % already defined.
% \end{LTXexample}  

% \subsection{Test FullName}

% \begin{LTXexample}
% \citecase{Ashcroft v. American Civil Liberties Union, 
%     535 U.S. 564 (2002)}
% \assert{FullName}{(\FullName{Ashcroft})}{(Ashcroft v. American Civil Liberties Union)}

% \edef\x{(\FullName{Ashcroft})}
% \assert{edef full name}{\x}{(Ashcroft v. American Civil Liberties Union)}

% Check full name w/o parens 

% \assert{FullName}{\FullName{Ashcroft}}{Ashcroft v. American Civil Liberties Union}

% \end{LTXexample}

% \subsection{Test cite}
% \begin{LTXexample}
%   \citecase{Ashcroft v. American Civil Liberties Union, 
%     535 U.S. 564 (2002)}
%   \citecase{Bashcroft v. American Civil Liberties Union, 
%     535 U.S. 564; 123 Mich 234 (2002)} % semicolon ; separate parallel citations 
%   {\makeatletter%
%     \let\CF\empty% disable emphasis of case name, we want all text
%     % \@bbCaseLongCite[\cccccc]{Ashcroft}{}
%     % \assert{long}{\c}{Ashcroft v. American Civil Liberties Union, 535 U.S. 564 (2002)}
%     \hrule 
%     \@bbCaseLongCite{Ashcroft}{2}
%   } 
% \end{LTXexample}

% \begin{LTXexample}
%   \citecase{Ashcroft v. American Civil Liberties Union, 
%     535 U.S. 564 (2002)}
%   \citecase{Bashcroft v. American Civil Liberties Union, 
%     535 U.S. 564; 123 Mich 234 (2002)} % separate parallel citations 
%                                 % with semicolon followed by a space 
%                                 % as shown. 
%   \citecase[Bashcroft2]{Bashcroft v.\@ American Civil Liberties Union, 
%     535 U.S. 564; 123 Mich 234 (2002)} % separate parallel citations 
%                                 % with semicolon followed by a space 
%                                 % as shown. 
%   {\makeatletter%
%     \let\CF\empty% this is used to emphasize text, disable it so we
%                  % can use assert
%     \hrule 
%     % \assert{2 cites}{\@bbCaseLongCite{Bashcroft}{}}{Bashcroft v. American Civil Liberties Union, 
%     % 535 U.S. 564; 123 Mich 234 (2002)}

%     \hrule 
%     \@bbCaseLongCite{Bashcroft}{2}
%     \@bbCaseLongCite{Bashcroft2}{2}
%   }
% \end{LTXexample}

% \begin{LTXexample}
%   \citecase{Ashcroft v. American Civil Liberties Union, 
%     535 U.S. 564 (2002)}
%   \citecase{Bashcroft v. American Civil Liberties Union, 
%     535 U.S. 564; 123 Mich 234 (2002)} % semicolon ; separate parallel citations 
%   {\makeatletter%
%     \let\CF\empty 
%     \hrule 
%     \@bbCaseLongCite{Bashcroft}{2; 3}
%   }
% \end{LTXexample}

% \begin{LTXexample}
%   \citecase{A v. B, 1 U.S. 2 (2002)}
%   \citecase{B v. B, 1 U.S. 2; 3 Mich App 4 (2002)}
%   \citecase{C v. B, 1 U.S. 2; 3 Mich App 4; 5 Mich App 6 (2002)}
%   {\makeatletter%
%     % \let\CF\empty 
%     \hrule 
%     \@bbCaseLongCite{A}{}\par 
%     \@bbCaseLongCite{A}{1}\par 
%     \@bbCaseLongCite{B}{}\par 
%     \@bbCaseLongCite{B}{1}\par 
%     \@bbCaseLongCite{B}{1; 2}\par 
%     \@bbCaseLongCite{C}{}\par 
%     \@bbCaseLongCite{C}{1}\par 
%     \@bbCaseLongCite{C}{1; 2}\par 
%     \@bbCaseLongCite{C}{1; 2; 3}\par 
%   }
% \end{LTXexample}

% \begin{LTXexample}
%   \citecase{A v. B, 1 Mich App 2 (2002)}
%   \citecase{B v. B, 1 U.S. 2; 3 Mich\,App 4 (2002)}
%   \citecase{C v. B, 1 U.S. 2; 3 Mich\,App 4; 5 Mich App 6 (2002)}
%   {\makeatletter%
%     % \let\CF\empty 
%     \hrule 
%     \@bbCaseShortCite{A}{}\par 
%     \@bbCaseShortCite{A}{1}\par 
%     \@bbCaseShortCite{B}{}\par 
%     \@bbCaseShortCite{B}{1}\par 
%     \@bbCaseShortCite{B}{1; 2}\par 
%     \@bbCaseShortCite{C}{}\par 
%     \@bbCaseShortCite{C}{1}\par 
%     \@bbCaseShortCite{C}{1; 2}\par 
%     \@bbCaseShortCite{C}{1; 2; 3}\par 
%   }

% \end{LTXexample}

% \begin{LTXexample}
%   \citecase{A v. B, \_\_  U.S. \_\_ (2002)}
%   {\makeatletter%
%     % \let\CF\empty 
%     \hrule 
%     \@bbCaseLongCite{A}{}\par  
%     \@bbCaseLongCite{A}{slip op at 1}\par 
%   }
% \end{LTXexample}

% \subsection{Testing cite and pincite}
% Testing cites and pincites cannot take place in LTXexample blocks.

%   \citecase{A v. B, 1 Mich App 2 (2002)}
%   \citecase{B v. B, 1 U.S. 2; 3 Mich\,App 4 (2002)}
%   \citecase{C v. B, 1 U.S. 2; 3 Mich\,App 4; 5 Mich App 6 (2002)}
%   Here are some pincites.
%  \pincite{A}{2} and \par
%   \pincite{B}{3} and \par
%   \pincite{A}{4} and \par 
%   \pincite{B}{2; 3} and \par
%   \pincite{A}{2} and \par
%   \pincite{C}{1} and 
%   \pincite{C}{2} and 
%   \pincite{C}{1; 3} and  
%   \pincite{C}{1; 2; 3} and  
%   And some cites  and \par
%   \cite{A} and \par
%   \cite{A} and \par

%   \cite{A} and \par
%   \cite{A} and \par
%   \cite{A} and \par
%   \cite{A} and \par
%   \cite{A} and \par

\end{document}