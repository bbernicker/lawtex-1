\providecommand{\documentclassflag}{}
\documentclass[12pt,\documentclassflag]{lawbrief} 

\usepackage[margin=1in]{geometry}
\usepackage{newcent,microtype}
\usepackage{setspace,xcolor}
\usepackage[hyperindex=false,linkbordercolor=white,pdfborder={0 0 0}]{hyperref}
\usepackage{test}
\usepackage{showexpl} % LTXexample blocks

%%Cases
% \makeandletter


\begin{document}

% \citecase{Ashcroft v. American Civil Liberties Union, 535 U.S. 564 (2002)}

\section{Testing the brief}

\subsection{Exploring ifcase}
\begin{LTXexample}
  \def\testcase#1{\ifcase#1 0: (#1)\or 1: (#1)\or 2: (#1)\or 3: (#1)\or 4: 
    (#1)\or 5: (#1)\else else (#1)\fi\par}

  \testcase{0}
  \testcase{1}
  \testcase{2}
  \testcase{3}
  \testcase{10}

  \edef\out{\testcase{0}}\out 
  \edef\out{\testcase{1}}\out 
  \edef\out{\testcase{2}}\out 
  \edef\out{\testcase{3}}\out 
  \edef\out{\testcase{10}}\out 

  \StrCount{}{1}[\c]\testcase\c 
  \StrCount{1}{1}[\c]\testcase\c 
  \StrCount{11}{1}[\c]\testcase\c 
  \StrCount{111}{1}[\c]\testcase\c 
  \StrCount{1111}{2}[\c]\testcase\c 
  \StrCount{11111}{2}[\c]\testcase\c 

\end{LTXexample}

\subsection{Test shortname}
\begin{LTXexample}
  shortcasename = (\deriveShortCaseName{Ashcroft v. American Civil
    Union})\par
  
  \deriveShortCaseName[\name]{Ashcroft v. American Civil Union}
  \assert{short name v.}{\name}{Ashcroft}\par

  \deriveShortCaseName[\name]{Ashcroft v American Civil Union}
  \assert{short name v}{\name}{Ashcroft}\par 
  
  \deriveShortCaseName[\name]{Ashcroft vs. American Civil Union}
  \assert{short name vs.}{\name}{Ashcroft vs. American Civil Union}\par
\end{LTXexample}
\subsection{Test parseCitations}
\begin{LTXexample}
  Basic test\par
  \let\reporters\empty\let\startpages\empty%
  parseCitations: (\parseCitations 123 reporter1 234; 345 reporter2 
    456; X X X ; ) parens should be empty\par 
  \assert{multiple reporters}{123 reporter1; 345 reporter2}{\reporters}\par 
  \assert{multiple start pages}{234; 456}{\startpages}\par 
  reporters: (\reporters)\par 
  start pages: (\startpages)\par 
  \hrule\par

  Can we do it again? Will reporters and pages reset?\par
  \let\reporters\empty\let\startpages\empty%
  parseCitations: (\parseCitations 123 reporter1 234; 345 reporter2 
    456; X X X ; ) parens should be empty\par
  \assert{repeat multiple reporters}{123 reporter1; 345 reporter2}{\reporters}\par 
  \assert{repeat multiple start pages}{234; 456}{\startpages}\par 
  reporters: (\reporters)\par
  start pages: (\startpages)\par
\end{LTXexample}
\begin{LTXexample} 
Can we parse citations stored in a macro?\par
  \let\reporters\empty\let\startpages\empty%
  \def\citations{123 reporter1 234; 345 reporter2 456}
  parseCitations: (\expandafter\parseCitations\citations; X X X; )
  parens should be empty\par
  \assert{multiple reporters}{123 reporter1; 345 reporter2}{\reporters}\par 
  \assert{multiple start pages}{234; 456}{\startpages}\par 
  reporters: (\reporters)\par 
  start pages: (\startpages)\par
\end{LTXexample}
\begin{LTXexample}
  Slip opinions\par
  \let\reporters\empty\let\startpages\empty%
  \parseCitations \_\_ Mich \_\_; X X X; 
  \assert{slip reporter}{\reporters}{\_\_ Mich}
  \assert{slip page}{\startpages}{\_\_}
  reporters = (\reporters)\par
  startpages = (\startpages)\par
  \end{LTXexample}
  \subsection{Test Citations}
  \begin{LTXexample}
    \def\testCitations#1#2#3#4#5#6#7{%
      % 1: shortname 2: reporters, 3: start pages, 
      % 4: pages, 5: slip suffix, 6: number of reporters, 7: number of 
      % pages.
      \edef\name{#1}%
      \edef\reporters{#2}%
      \edef\startpages{#3}%
      \edef\pages{#4}%
      \edef\slipsuffix{#5}%
      \edef\numberOfReporters{#6}%
      \edef\numberOfPages{#7}%
      \def\check##1##2##3##4##5##6##7{%
        \par check 1(##1) 2(##2) 3(##3) 4(##4) 5(##5) 6(##6) 7(##7)\par%
        \par testing\par%
        \assert{name}{\name}{##1}%
        \assert{reporters}{\reporters}{##2}%
        \assert{startpages}{\startpages}{##3}%
        \assert{pages}{\pages}{##4}%
        \assert{slipsuffix}{\slipsuffix}{##5}%
        \assert{numberOfReporters}{\numberOfReporters}{##6}%
        \assert{numberOfPages}{\numberOfPages}{##7}%
      }%
      testCitations 1(#1) 2(#2) 3(#3) 4(#4) 5(#5) 6(#6) 7(#7)\par%
      \bbCaseCitationEval{\name}{\pages}{\check}%
    }%

    \citecase{A1 v. B, 1 U.S. 2 (2002)}\par
    % \citecase{B1 v. B, 1 U.S. 2; 3 Mich App 4 (2002)}
    % \citecase{C1 v. B, 1 U.S. 2; 3 Mich App 4; 5 Mich App 6 (2002)}
    \testCitations{A1}{1 U.S.}{2}{3}{\empty}{1}{1}\par
    % \testCitations{B1}{1 U.S.; 3 Mich App}{2; 4]{3}{}{2}{1}
    % \testCitations{C1}{1 U.S.; 3 Mich App; 5 Mich App}{2; 4; 6]{3}{}{3}{1}
 
  \end{LTXexample}
  \subsection{Test case parsing}
\begin{LTXexample}
  \def\cont#1#2#3#4#5{%
    \par%
    short name: (#1)\par%
    case name: (#2)\par%
    reporters: (#3)\par%
    start pages: (#4)\par%
    parenthetical: (#5)\par%
    \vspace{5mm}
  }
  \long\def\checkparse#1#2#3#4#5#6#7{{%1,2: args to parseCase, 3-7:
      % expected
      % in extra blocks to avoid contamination
      \long\def\check##1##2##3##4##5{%
        \assert{short name}{##1}{#3}\par%
        \assert{case name}{##2}{#4}\par%
        \assert{reporters}{##3}{#5}\par%
        \assert{start pages}{##4}{#6}\par%
        \assert{parenthetical}{##5}{#7}\par%
      }% end check
      \long\def\checkShortname##1##2##3##4##5{%
        \assert{derived short name}{##1}{#3}\par%
        \vspace{5mm}%
      }% end checkShortname
      parsing with arguments (#1) and (#2)\par
      \parseCase[#1]{#2}{\check}%
      \parseCase{#2}{\checkShortname}%
    }}%end checkparse
  
  \checkparse{Ashcroft}{Ashcroft v. American Civil Liberties Union, 
    535 U.S. 564 (2002)}{Ashcroft}{Ashcroft v. American Civil 
    Liberties Union}{535 U.S.}{564}{(2002)}\par 
  \checkparse{Ashcroft}{Ashcroft v. American Civil Liberties Union, 535 U.S. 564; 
    123 nw2d 456 (2002)}{Ashcroft}{Ashcroft v. American Civil 
    Liberties Union}{535 U.S.; 123 nw2d}{564; 456}{(2002)}

\end{LTXexample}

\subsection{Test citecase}

\begin{LTXexample}
  \citecase{x v y, 12 ny2d 34 (2015)}
  \citecase[X]{X v Y, 12 ny2d 34 (2015)}
  % \citecase[X]{X v Y, 12 ny2d 34 (2015)} % throws error, X is
  % already defined.
\end{LTXexample}  

\subsection{Test FullName}

\begin{LTXexample}
\citecase{Ashcroft v. American Civil Liberties Union, 
    535 U.S. 564 (2002)}
\assert{FullName}{(\FullName{Ashcroft})}{(Ashcroft v. American Civil Liberties Union)}

\edef\x{(\FullName{Ashcroft})}
\assert{edef full name}{\x}{(Ashcroft v. American Civil Liberties Union)}

Check full name w/o parens 

\assert{FullName}{\FullName{Ashcroft}}{Ashcroft v. American Civil Liberties Union}

\end{LTXexample}

\subsection{Test cite}
\begin{LTXexample}
  \citecase{Ashcroft v. American Civil Liberties Union, 
    535 U.S. 564 (2002)}
  \citecase{Bashcroft v. American Civil Liberties Union, 
    535 U.S. 564; 123 Mich 234 (2002)} % semicolon ; separate parallel citations 
  {\makeatletter%
    \let\CF\empty% disable emphasis of case name, we want all text
    % \@bbCaseLongCite[\cccccc]{Ashcroft}{}
    % \assert{long}{\c}{Ashcroft v. American Civil Liberties Union, 535 U.S. 564 (2002)}
    \hrule 
    \@bbCaseLongCite{Ashcroft}{2}
  } 
\end{LTXexample}

\begin{LTXexample}
  \citecase{Ashcroft v. American Civil Liberties Union, 
    535 U.S. 564 (2002)}
  \citecase{Bashcroft v. American Civil Liberties Union, 
    535 U.S. 564; 123 Mich 234 (2002)} % separate parallel citations 
                                % with semicolon followed by a space 
                                % as shown. 
  \citecase[Bashcroft2]{Bashcroft v.\@ American Civil Liberties Union, 
    535 U.S. 564; 123 Mich 234 (2002)} % separate parallel citations 
                                % with semicolon followed by a space 
                                % as shown. 
  {\makeatletter%
    \let\CF\empty% this is used to emphasize text, disable it so we
                 % can use assert
    \hrule 
    % \assert{2 cites}{\@bbCaseLongCite{Bashcroft}{}}{Bashcroft v. American Civil Liberties Union, 
    % 535 U.S. 564; 123 Mich 234 (2002)}

    \hrule 
    \@bbCaseLongCite{Bashcroft}{2}
    \@bbCaseLongCite{Bashcroft2}{2}
  }
\end{LTXexample}

\begin{LTXexample}
  \citecase{Ashcroft v. American Civil Liberties Union, 
    535 U.S. 564 (2002)}
  \citecase{Bashcroft v. American Civil Liberties Union, 
    535 U.S. 564; 123 Mich 234 (2002)} % semicolon ; separate parallel citations 
  {\makeatletter%
    \let\CF\empty 
    \hrule 
    \@bbCaseLongCite{Bashcroft}{2; 3}
  }
\end{LTXexample}

\begin{LTXexample}
  \citecase{A v. B, 1 U.S. 2 (2002)}
  \citecase{B v. B, 1 U.S. 2; 3 Mich App 4 (2002)}
  \citecase{C v. B, 1 U.S. 2; 3 Mich App 4; 5 Mich App 6 (2002)}
  {\makeatletter%
    % \let\CF\empty 
    \hrule 
    \@bbCaseLongCite{A}{}\par 
    \@bbCaseLongCite{A}{1}\par 
    \@bbCaseLongCite{B}{}\par 
    \@bbCaseLongCite{B}{1}\par 
    \@bbCaseLongCite{B}{1; 2}\par 
    \@bbCaseLongCite{C}{}\par 
    \@bbCaseLongCite{C}{1}\par 
    \@bbCaseLongCite{C}{1; 2}\par 
    \@bbCaseLongCite{C}{1; 2; 3}\par 
  }
\end{LTXexample}

\begin{LTXexample}
  \citecase{A v. B, 1 Mich App 2 (2002)}
  \citecase{B v. B, 1 U.S. 2; 3 Mich\,App 4 (2002)}
  \citecase{C v. B, 1 U.S. 2; 3 Mich\,App 4; 5 Mich App 6 (2002)}
  {\makeatletter%
    % \let\CF\empty 
    \hrule 
    \@bbCaseShortCite{A}{}\par 
    \@bbCaseShortCite{A}{1}\par 
    \@bbCaseShortCite{B}{}\par 
    \@bbCaseShortCite{B}{1}\par 
    \@bbCaseShortCite{B}{1; 2}\par 
    \@bbCaseShortCite{C}{}\par 
    \@bbCaseShortCite{C}{1}\par 
    \@bbCaseShortCite{C}{1; 2}\par 
    \@bbCaseShortCite{C}{1; 2; 3}\par 
  }

\end{LTXexample}

\begin{LTXexample}
  \citecase{A v. B, \_\_  U.S. \_\_ (2002)}
  {\makeatletter%
    % \let\CF\empty 
    \hrule 
    \@bbCaseLongCite{A}{}\par  
    \@bbCaseLongCite{A}{slip op at 1}\par 
  }
\end{LTXexample}

\subsection{Testing cite and pincite}
Testing cites and pincites cannot take place in LTXexample blocks.

  \citecase{A v. B, 1 Mich App 2 (2002)}
  \citecase{B v. B, 1 U.S. 2; 3 Mich\,App 4 (2002)}
  \citecase{C v. B, 1 U.S. 2; 3 Mich\,App 4; 5 Mich App 6 (2002)}
  Here are some pincites.
  \pincite{A}{2} and \par
  \pincite{B}{3} and \par
  \pincite{A}{4} and \par 
  \pincite{B}{2; 3} and \par
  \pincite{A}{2} and \par
  \pincite{C}{1} and 
  \pincite{C}{2} and 
  \pincite{C}{1; 3} and  
  \pincite{C}{1; 2; 3} and 
  And some cites  and \par
  \cite{A} and \par
  \cite{A} and \par
  \cite{A} and \par
  \cite{A} and \par
  \cite{A} and \par
  \cite{A} and \par
  \cite{A} and \par
 \end{document}